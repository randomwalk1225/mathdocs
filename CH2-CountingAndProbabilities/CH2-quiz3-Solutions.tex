\documentclass[11pt,a4paper]{article}
\usepackage{fontspec}
\usepackage{amsmath,amssymb}
\usepackage{enumitem}
\usepackage{geometry}
\usepackage{titlesec}
\usepackage{xcolor}
\usepackage{booktabs}

\geometry{margin=2cm}

% Font settings
\setmainfont{Times New Roman}
\setsansfont{Arial}

% Color definitions
\definecolor{problemcolor}{RGB}{0,51,102}
\definecolor{solutioncolor}{RGB}{0,102,51}
\definecolor{answercolor}{RGB}{153,0,0}

\title{\textbf{TMUA Chapter 2 - Quiz 3:\\Counting and Probabilities Supplements S02\\(With Solutions)}}
\author{}
\date{}

\begin{document}

\maketitle

\noindent\textbf{Time Allowed:} 90 minutes\\
\textbf{Number of Questions:} 20\\
\textbf{Difficulty:} $\star\star\star\circ$

\vspace{1em}
\hrule
\vspace{1em}

\section{Supplement Questions}

\subsection{SQ1}

\noindent\textcolor{problemcolor}{\textbf{PROBLEM:}}

Anne, Bert, Clare, Derek and Emily are planning to play a game for which they need to divide themselves into three teams. Each team must have at least one member. The number of different ways they can do this is

\begin{enumerate}[label=(\Alph*)]
\item 10
\item 15
\item 25
\item 30
\end{enumerate}

\noindent\textcolor{solutioncolor}{\textbf{SOLUTION:}}

We need to partition 5 people into 3 non-empty groups. This is a Stirling number of the second kind problem, denoted $S(5,3)$.

The possible group sizes are:
\begin{itemize}
\item 3, 1, 1: One team of 3, two teams of 1
\item 2, 2, 1: Two teams of 2, one team of 1
\end{itemize}

\textbf{Case 1: Groups of size (3, 1, 1)}

Choose 3 people for the first team: $\binom{5}{3} = 10$ ways

The remaining 2 people form two singleton teams. Since the two singleton teams are indistinguishable (unlabeled teams), we have 10 ways.

\textbf{Case 2: Groups of size (2, 2, 1)}

Choose 2 people for first pair: $\binom{5}{2} = 10$

Choose 2 people from remaining 3: $\binom{3}{2} = 3$

The last person forms a singleton.

Since two pairs are indistinguishable: $(10 \times 3) / 2! = 30/2 = 15$ ways

Total = $10 + 15 = 25$ ways

\noindent\textcolor{answercolor}{\textbf{ANSWER: (C) 25}}

\vspace{1em}
\hrule
\vspace{1em}

\subsection{SQ2}

\noindent\textcolor{problemcolor}{\textbf{PROBLEM:}}

The faces of a cube are coloured red or blue. Exactly three are red and three are blue. The number of distinguishable cubes that can be produced (allowing the cube to be turned around) is?

\begin{enumerate}[label=(\Alph*)]
\item 2
\item 4
\item 6
\item 20
\end{enumerate}

\noindent\textcolor{solutioncolor}{\textbf{SOLUTION:}}

We need to count distinct colorings of a cube with 3 red faces and 3 blue faces, where two colorings are the same if one can be rotated into the other.

Consider how the three red faces can be arranged:

\textbf{Type 1:} Three faces sharing a common vertex

All three red faces meet at one corner. This is one distinct coloring.

\textbf{Type 2:} Three faces forming a band

Three faces that do not all share a common vertex form a strip around the cube. This is another distinct coloring.

After accounting for rotational symmetry, there are exactly 2 distinguishable cubes.

\noindent\textcolor{answercolor}{\textbf{ANSWER: (A) 2}}

\vspace{1em}
\hrule
\vspace{1em}

\subsection{SQ3}

\noindent\textcolor{problemcolor}{\textbf{PROBLEM:}}

A grid of size 3 cm $\times$ 5 cm is drawn, ruled at 1 cm intervals. The number of squares that can be drawn using the grid is

\begin{enumerate}[label=(\Alph*)]
\item 15
\item 18
\item 26
\item 37
\end{enumerate}

\noindent\textcolor{solutioncolor}{\textbf{SOLUTION:}}

The grid has 4 horizontal lines and 6 vertical lines (creating $3 \times 5 = 15$ unit squares).

We can form squares of various sizes:

\textbf{$1\times 1$ squares:}

Number of positions: $3 \times 5 = 15$

\textbf{$2\times 2$ squares:}

Horizontal positions: $3 - 1 = 2$

Vertical positions: $5 - 1 = 4$

Total: $2 \times 4 = 8$

\textbf{$3\times 3$ squares:}

Horizontal positions: $3 - 2 = 1$

Vertical positions: $5 - 2 = 3$

Total: $1 \times 3 = 3$

\textbf{$4\times 4$ squares or larger:}

Not possible since the grid is only 3 cm tall.

Total = $15 + 8 + 3 = 26$

\noindent\textcolor{answercolor}{\textbf{ANSWER: (C) 26}}

\vspace{1em}
\hrule
\vspace{1em}

\subsection{SQ4}

\noindent\textcolor{problemcolor}{\textbf{PROBLEM:}}

A cube painted black is cut into 125 identical cubes. How many of them are not painted at all?

\begin{enumerate}[label=(\Alph*)]
\item 21
\item 25
\item 27
\item 30
\end{enumerate}

\noindent\textcolor{solutioncolor}{\textbf{SOLUTION:}}

If the large cube is cut into $125 = 5^3$ identical small cubes, then each edge is divided into 5 equal parts.

The small cubes can be classified by position:
\begin{itemize}
\item \textbf{Corner cubes}: 8 cubes (3 faces painted)
\item \textbf{Edge cubes} (not corners): $12 \times 3 = 36$ cubes (2 faces painted)
\item \textbf{Face cubes} (not on edges): $6 \times 9 = 54$ cubes (1 face painted)
\item \textbf{Interior cubes}: Not painted
\end{itemize}

Interior cubes (not painted):

The unpainted cubes form a smaller cube inside, with edge length $(5-2) = 3$.

Number of unpainted cubes = $3^3 = 27$

\noindent\textcolor{answercolor}{\textbf{ANSWER: (C) 27}}

\vspace{1em}
\hrule
\vspace{1em}

\subsection{SQ5}

\noindent\textcolor{problemcolor}{\textbf{PROBLEM:}}

90 people enter a maze. At each junction a third will go left and two thirds will go right. After three such junctions, what is the most likely combination of turns people will have taken?

\begin{enumerate}[label=(\Alph*)]
\item Gone right three times
\item Gone left three times
\item Gone right twice and once left
\item Gone twice left and once right
\item It is impossible to tell
\end{enumerate}

\noindent\textcolor{solutioncolor}{\textbf{SOLUTION:}}

At each junction: $P(\text{left}) = 1/3$, $P(\text{right}) = 2/3$

For three junctions, we have outcomes with probabilities:
\begin{itemize}
\item \textbf{RRR} (right 3 times): $(2/3)^3 = 8/27 \approx 0.296$
\item \textbf{RRL, RLR, LRR} (right 2, left 1): $3 \times (2/3)^2 \times (1/3) = 3 \times 4/9 \times 1/3 = 12/27 = 4/9 \approx 0.444$
\item \textbf{RLL, LRL, LLR} (right 1, left 2): $3 \times (2/3) \times (1/3)^2 = 3 \times 2/3 \times 1/9 = 6/27 = 2/9 \approx 0.222$
\item \textbf{LLL} (left 3 times): $(1/3)^3 = 1/27 \approx 0.037$
\end{itemize}

The probability for ``right twice and left once'' = $12/27 \approx 0.444$

This is the highest probability.

\noindent\textcolor{answercolor}{\textbf{ANSWER: (C) Gone right twice and once left}}

\vspace{1em}
\hrule
\vspace{1em}

\subsection{SQ6}

\noindent\textcolor{problemcolor}{\textbf{PROBLEM:}}

A bag contains $b$ blue balls and $r$ red balls. If two balls are picked at random and removed from the bag, what is the probability $P$ that they are different colours?

\begin{enumerate}[label=(\Alph*)]
\item $\dfrac{2br}{(b + r)(b + r - 1)}$
\item $\dfrac{br}{(b + r)(b + r - 1)}$
\item $\dfrac{br}{(b + r)^2}$
\item $\dfrac{2br}{(b + r)^2}$
\item $2br$
\end{enumerate}

\noindent\textcolor{solutioncolor}{\textbf{SOLUTION:}}

Total balls = $b + r$

The probability of picking two balls of different colors can happen in two ways:

1. Pick blue first, then red: $P(BR) = \dfrac{b}{b+r} \times \dfrac{r}{b+r-1}$

2. Pick red first, then blue: $P(RB) = \dfrac{r}{b+r} \times \dfrac{b}{b+r-1}$

$P(\text{different colors}) = P(BR) + P(RB)$

$= \dfrac{br}{(b+r)(b+r-1)} + \dfrac{rb}{(b+r)(b+r-1)}$

$= \dfrac{2br}{(b+r)(b+r-1)}$

\noindent\textcolor{answercolor}{\textbf{ANSWER: (A) $\dfrac{2br}{(b + r)(b + r - 1)}$}}

\vspace{1em}
\hrule
\vspace{1em}

\subsection{SQ7}

\noindent\textcolor{problemcolor}{\textbf{PROBLEM:}}

We wish to represent integer numbers by using our ten fingers. A finger is assumed to be either stretched out or curled up. How many different integers can we represent with our fingers?

\begin{enumerate}[label=(\Alph*)]
\item 10
\item 512
\item 1000
\item 20
\item 1024
\end{enumerate}

\noindent\textcolor{solutioncolor}{\textbf{SOLUTION:}}

Each finger has 2 possible states: stretched out or curled up.

With 10 fingers, the total number of different configurations is:

$2 \times 2 \times 2 \times 2 \times 2 \times 2 \times 2 \times 2 \times 2 \times 2 = 2^{10} = 1024$

Each configuration can represent a different integer, so we can represent 1024 different integers (typically 0 through 1023 in binary).

\noindent\textcolor{answercolor}{\textbf{ANSWER: (E) 1024}}

\vspace{1em}
\hrule
\vspace{1em}

\subsection{SQ8}

\noindent\textcolor{problemcolor}{\textbf{PROBLEM:}}

Ten students need to complete their compulsory practicals for their high school examinations as detailed in the table below:

\begin{center}
\begin{tabular}{cc}
\toprule
No. of students & No. of different practicals to complete \\
\midrule
2 & 1 \\
4 & 2 \\
4 & 3 \\
\bottomrule
\end{tabular}
\end{center}

The school only has one laboratory in which several different experiments can be set up simultaneously. A maximum of six students are allowed in the school's laboratory for a lesson. Each practical takes one lesson. What is the minimum number of lessons required to complete all the practicals?

\begin{enumerate}[label=(\Alph*)]
\item 3
\item 4
\item 5
\item 6
\item 10
\end{enumerate}

\noindent\textcolor{solutioncolor}{\textbf{SOLUTION:}}

Total practical sessions needed:
\begin{itemize}
\item 2 students $\times$ 1 practical each = 2 sessions
\item 4 students $\times$ 2 practicals each = 8 sessions
\item 4 students $\times$ 3 practicals each = 12 sessions
\item Total = $2 + 8 + 12 = 22$ student-sessions
\end{itemize}

Since at most 6 students can be in the lab per lesson:

Minimum lessons = $\lceil 22/6 \rceil = \lceil 3.67 \rceil = 4$ lessons

Let's verify a schedule:
\begin{itemize}
\item Lesson 1: All 4 ``three-practical'' students + 2 ``two-practical'' students = 6 students
\item Lesson 2: Same 4 ``three-practical'' students + 2 ``two-practical'' students = 6 students
\item Lesson 3: Same 4 ``three-practical'' students + 2 ``one-practical'' students = 6 students
\item Lesson 4: Remaining 2 ``two-practical'' students complete their second practical = 2 students
\end{itemize}

This works! Minimum = 4 lessons.

\noindent\textcolor{answercolor}{\textbf{ANSWER: (B) 4}}

\vspace{1em}
\hrule
\vspace{1em}

\subsection{SQ9}

\noindent\textcolor{problemcolor}{\textbf{PROBLEM:}}

To get to work, Sylvie first catches a bus and then catches a train.

\begin{itemize}
\item The probability that the bus is on time is 0.6.
\item The probability that the bus is late is 0.4.
\item If the bus is on time, then the probability that she will catch the train is 0.8.
\item If the bus is late, then the probability that she will catch the train is 0.6.
\end{itemize}

Given that Sylvie catches the train, what is the probability that the bus was on time?

\begin{enumerate}[label=(\Alph*)]
\item $1/3$
\item $12/25$
\item $2/5$
\item $3/5$
\item $2/3$
\item $18/25$
\item $6/7$
\end{enumerate}

\noindent\textcolor{solutioncolor}{\textbf{SOLUTION:}}

Let $B$ = bus on time, $T$ = catches train

Given:
\begin{itemize}
\item $P(B) = 0.6$, $P(B') = 0.4$
\item $P(T|B) = 0.8$
\item $P(T|B') = 0.6$
\end{itemize}

We need $P(B|T)$ using Bayes' theorem:

$P(B|T) = \dfrac{P(T|B) \times P(B)}{P(T)}$

First, find $P(T)$:

$P(T) = P(T|B) \times P(B) + P(T|B') \times P(B')$

$= 0.8 \times 0.6 + 0.6 \times 0.4$

$= 0.48 + 0.24 = 0.72$

Now:

$P(B|T) = \dfrac{0.8 \times 0.6}{0.72} = \dfrac{0.48}{0.72} = \dfrac{48}{72} = \dfrac{2}{3}$

\noindent\textcolor{answercolor}{\textbf{ANSWER: (E) $2/3$}}

\vspace{1em}
\hrule
\vspace{1em}

\subsection{SQ10}

\noindent\textcolor{problemcolor}{\textbf{PROBLEM:}}

I have two six-sided dice, each with faces numbered from 1 to 6. One of the dice is fair, but the other is not; it will land on numbers 1 to 5 with equal probability, but lands on 6 with a different probability.

When I roll the dice the probability that I get a total of 12 is $1/18$.

What is the probability that I get a total of 2 when I roll the dice?

\begin{enumerate}[label=(\Alph*)]
\item $1/72$
\item $1/45$
\item $1/36$
\item $1/18$
\item $1/9$
\end{enumerate}

\noindent\textcolor{solutioncolor}{\textbf{SOLUTION:}}

Let the fair die be $F$ and the unfair die be $U$.

For $F$: $P(i) = 1/6$ for $i = 1, 2, 3, 4, 5, 6$

For $U$: Let $P(1) = P(2) = P(3) = P(4) = P(5) = p$, and $P(6) = q$

Since probabilities sum to 1: $5p + q = 1$

To get a sum of 12, we need both dice to show 6:

$P(\text{sum} = 12) = P(F=6) \times P(U=6) = (1/6) \times q = 1/18$

Therefore: $q = 1/3$

From $5p + q = 1$:

$5p + 1/3 = 1$

$5p = 2/3$

$p = 2/15$

To get a sum of 2, we need both dice to show 1:

$P(\text{sum} = 2) = P(F=1) \times P(U=1) = (1/6) \times (2/15) = 2/90 = 1/45$

\noindent\textcolor{answercolor}{\textbf{ANSWER: (B) $1/45$}}

\vspace{1em}
\hrule
\vspace{1em}

\subsection{SQ11}

\noindent\textcolor{problemcolor}{\textbf{PROBLEM:}}

The ratio of the number of boys to the number of girls in a class is 1:3

The number of boys in the class is $n$.

Two students are chosen at random from the class.

The probability that both the students are boys is $p$.

Which one of the following is a correct expression for $n$, the number of boys in the class?

\begin{enumerate}[label=(\Alph*)]
\item $n = \dfrac{3p - 1}{9p - 1}$
\item $n = \dfrac{3p + 1}{9p - 1}$
\item $n = \dfrac{1}{1 - 9p}$
\item $n = \dfrac{1}{9p - 1}$
\item $n = \dfrac{4p - 1}{16p - 1}$
\item $n = \dfrac{4p + 1}{16p - 1}$
\item $n = \dfrac{1}{1 - 16p}$
\item $n = \dfrac{1}{16p - 1}$
\end{enumerate}

\noindent\textcolor{solutioncolor}{\textbf{SOLUTION:}}

Number of boys = $n$

Ratio boys:girls = 1:3, so number of girls = $3n$

Total students = $n + 3n = 4n$

Probability that both students are boys:

$p = \dfrac{\binom{n}{2}}{\binom{4n}{2}}$

$p = \dfrac{n(n-1)/2}{4n(4n-1)/2}$

$p = \dfrac{n(n-1)}{4n(4n-1)}$

$p = \dfrac{n-1}{4(4n-1)}$

Now solve for $n$:

$p \times 4(4n-1) = n - 1$

$4p(4n-1) = n - 1$

$16pn - 4p = n - 1$

$16pn - n = 4p - 1$

$n(16p - 1) = 4p - 1$

$n = \dfrac{4p - 1}{16p - 1}$

\noindent\textcolor{answercolor}{\textbf{ANSWER: (E) $n = \dfrac{4p - 1}{16p - 1}$}}

\vspace{1em}
\hrule
\vspace{1em}

\subsection{SQ12}

\noindent\textcolor{problemcolor}{\textbf{PROBLEM:}}

A bag contains only $n$ red balls and $2n$ green balls.

One ball is picked and its colour recorded. It is then put back in the bag, and an additional ball of the same colour is added to the bag.

A second ball is then picked.

What is the probability that the two balls picked are not the same colour?

\begin{enumerate}[label=(\Alph*)]
\item $\dfrac{2n}{3(3n - 1)}$
\item $\dfrac{4n}{3(3n - 1)}$
\item $\dfrac{5n}{3(3n - 1)}$
\item $\dfrac{5n - 3}{3(3n - 1)}$
\item $\dfrac{2n}{3(3n + 1)}$
\item $\dfrac{4n}{3(3n + 1)}$
\item $\dfrac{5n}{3(3n + 1)}$
\item $\dfrac{5n + 3}{3(3n + 1)}$
\end{enumerate}

\noindent\textcolor{solutioncolor}{\textbf{SOLUTION:}}

Initial: $n$ red, $2n$ green, total = $3n$ balls

\textbf{Case 1: First ball red, second ball green}

$P(\text{first red}) = n/(3n) = 1/3$

After replacing red and adding one red: $(n+1)$ red, $2n$ green, total = $3n+1$

$P(\text{second green} | \text{first red}) = 2n/(3n+1)$

$P(RG) = (1/3) \times [2n/(3n+1)] = 2n / [3(3n+1)]$

\textbf{Case 2: First ball green, second ball red}

$P(\text{first green}) = 2n/(3n) = 2/3$

After replacing green and adding one green: $n$ red, $(2n+1)$ green, total = $3n+1$

$P(\text{second red} | \text{first green}) = n/(3n+1)$

$P(GR) = (2/3) \times [n/(3n+1)] = 2n / [3(3n+1)]$

$P(\text{different colors}) = P(RG) + P(GR)$

$= \dfrac{2n}{3(3n+1)} + \dfrac{2n}{3(3n+1)} = \dfrac{4n}{3(3n+1)}$

\noindent\textcolor{answercolor}{\textbf{ANSWER: (F) $\dfrac{4n}{3(3n + 1)}$}}

\vspace{1em}
\hrule
\vspace{1em}

\subsection{SQ13}

\noindent\textcolor{problemcolor}{\textbf{PROBLEM:}}

A bag only contains $2n$ blue balls and $n$ red balls. All the balls are identical apart from colour.

One ball is randomly selected and not replaced. A second ball is then randomly selected.

What is the probability that at least one of the selected balls is red?

\begin{enumerate}[label=(\Alph*)]
\item $\dfrac{n - 1}{3(3n - 1)}$
\item $\dfrac{3n - 1}{3(3n - 1)}$
\item $\dfrac{4n - 2}{3(3n - 1)}$
\item $\dfrac{4n}{3(3n - 1)}$
\item $\dfrac{5n - 1}{3(3n - 1)}$
\item $\dfrac{5n - 5}{3(3n - 1)}$
\end{enumerate}

\noindent\textcolor{solutioncolor}{\textbf{SOLUTION:}}

Total balls = $2n + n = 3n$

It's easier to use the complement: $P(\text{at least one red}) = 1 - P(\text{both blue})$

$P(\text{both blue}) = P(\text{first blue}) \times P(\text{second blue} | \text{first blue})$

$= \dfrac{2n}{3n} \times \dfrac{2n-1}{3n-1}$

$= \dfrac{2n(2n-1)}{3n(3n-1)}$

$= \dfrac{4n^2 - 2n}{3n(3n-1)}$

$= \dfrac{2n(2n-1)}{3n(3n-1)}$

$= \dfrac{2(2n-1)}{3(3n-1)}$

$= \dfrac{4n-2}{3(3n-1)}$

$P(\text{at least one red}) = 1 - \dfrac{4n-2}{3(3n-1)}$

$= \dfrac{3(3n-1) - (4n-2)}{3(3n-1)}$

$= \dfrac{9n - 3 - 4n + 2}{3(3n-1)}$

$= \dfrac{5n - 1}{3(3n-1)}$

\noindent\textcolor{answercolor}{\textbf{ANSWER: (E) $\dfrac{5n - 1}{3(3n - 1)}$}}

\vspace{1em}
\hrule
\vspace{1em}

\subsection{SQ14}

\noindent\textcolor{problemcolor}{\textbf{PROBLEM:}}

Two identical fair six-sided dice each have their faces numbered from 1 to 6, with one number on each face.

Both dice are thrown, and the number on each of the dice is recorded.

They are then both thrown again, and the number on each of the dice is recorded.

What is the probability that at least one of the four recorded numbers is even?

\begin{enumerate}[label=(\Alph*)]
\item $1/4$
\item $1/2$
\item $9/16$
\item $3/4$
\item $15/16$
\end{enumerate}

\noindent\textcolor{solutioncolor}{\textbf{SOLUTION:}}

Use the complement: $P(\text{at least one even}) = 1 - P(\text{all four odd})$

For a single die: $P(\text{odd}) = 3/6 = 1/2$, $P(\text{even}) = 3/6 = 1/2$

For four independent rolls:

$P(\text{all four odd}) = (1/2) \times (1/2) \times (1/2) \times (1/2) = 1/16$

$P(\text{at least one even}) = 1 - 1/16 = 15/16$

\noindent\textcolor{answercolor}{\textbf{ANSWER: (E) $15/16$}}

\vspace{1em}
\hrule
\vspace{1em}

\subsection{SQ15}

\noindent\textcolor{problemcolor}{\textbf{PROBLEM:}}

Eight people are sitting around a circular table, each holding a fair coin. All eight people flip their coins and those who flip heads stand while those who flip tails remain seated. What is the probability that no two adjacent people will stand?

\begin{enumerate}[label=(\Alph*)]
\item $47/256$
\item $3/16$
\item $49/256$
\item $25/128$
\item $51/256$
\end{enumerate}

\noindent\textcolor{solutioncolor}{\textbf{SOLUTION:}}

Total possible outcomes = $2^8 = 256$

We need to count arrangements where no two adjacent people stand (flip heads).

Let $a_n$ be the number of valid arrangements for $n$ people in a circle where no two adjacent stand.

For a circle of 8 people, using the recurrence relation for circular arrangements where no two adjacent positions are selected, the number of valid arrangements is given by Lucas numbers.

For $n=8$: valid arrangements = 47

$P(\text{no two adjacent stand}) = 47/256$

\noindent\textcolor{answercolor}{\textbf{ANSWER: (A) $47/256$}}

\vspace{1em}
\hrule
\vspace{1em}

\subsection{SQ16}

\noindent\textcolor{problemcolor}{\textbf{PROBLEM:}}

A choir director must select a group of singers from among his 6 tenors and 8 basses. The only requirements are that the difference between the number of tenors and basses must be a multiple of 4, and the group must have at least one singer. Let $N$ be the number of groups that can be selected. What is the remainder when $N$ is divided by 100?

\begin{enumerate}[label=(\Alph*)]
\item 47
\item 48
\item 83
\item 95
\item 96
\end{enumerate}

\noindent\textcolor{solutioncolor}{\textbf{SOLUTION:}}

Let $t$ = number of tenors ($0 \le t \le 6$), $b$ = number of basses ($0 \le b \le 8$)

Condition: $|t - b| \equiv 0 \pmod{4}$, which means $t - b \equiv 0 \pmod{4}$

Also: $t + b \ge 1$ (at least one singer)

Valid pairs when $t - b \in \{0, \pm 4, \pm 8\}$:

\textbf{When $t - b = 0$:} $(1,1), (2,2), (3,3), (4,4), (5,5), (6,6)$ — 6 pairs

\textbf{When $t - b = 4$:} $(4,0), (5,1), (6,2)$ — 3 pairs

\textbf{When $t - b = -4$:} $(0,4), (1,5), (2,6), (3,7), (4,8)$ — 5 pairs

\textbf{When $t - b = -8$:} $(0,8)$ — 1 pair

Total pairs: $6 + 3 + 5 + 1 = 15$

For each pair $(t, b)$, number of ways = $\binom{6}{t} \times \binom{8}{b}$

Computing all terms:

$N = \sum \binom{6}{t} \times \binom{8}{b}$ over all valid pairs

$= 48 + 420 + 1120 + 1050 + 336 + 28 + 15 + 48 + 28 + 70 + 336 + 420 + 160 + 15 + 1 = 4095$

$N \bmod 100 = 4095 \bmod 100 = 95$

\noindent\textcolor{answercolor}{\textbf{ANSWER: (D) 95}}

\vspace{1em}
\hrule
\vspace{1em}

\subsection{SQ17}

\noindent\textcolor{problemcolor}{\textbf{PROBLEM:}}

How many 15-letter arrangements of 5 A's, 5 B's and 5 C's have no A's in the first 5 letters, no B's in the next 5 letters and no C's in the last 5 letters?

\begin{enumerate}[label=(\Alph*)]
\item $\displaystyle\sum_{k=0}^{5} \binom{5}{k}^3$
\item $3^5 \cdot 2^5$
\item $2^{15}$
\item $\dfrac{15!}{(5!)^3}$
\item $3^{15}$
\end{enumerate}

\noindent\textcolor{solutioncolor}{\textbf{SOLUTION:}}

Divide the 15 positions into three groups:
\begin{itemize}
\item Positions 1-5: can have B's or C's (not A's)
\item Positions 6-10: can have A's or C's (not B's)
\item Positions 11-15: can have A's or B's (not C's)
\end{itemize}

Let $k$ be the number of B's in positions 1-5. Then:
\begin{itemize}
\item Positions 1-5 have $k$ B's and $(5-k)$ C's
\item Positions 6-10 have $k$ C's and $(5-k)$ A's
\item Positions 11-15 have $(5-k)$ B's and $k$ A's
\end{itemize}

For positions 1-5: Choose $k$ positions for B's: $\binom{5}{k}$

For positions 6-10: Choose positions for $k$ C's among 5 positions: $\binom{5}{k}$

For positions 11-15: Choose positions for $(5-k)$ B's among 5 positions: $\binom{5}{5-k} = \binom{5}{k}$

Total for each $k$: $\binom{5}{k} \times \binom{5}{k} \times \binom{5}{k} = \binom{5}{k}^3$

Sum over all $k$ from 0 to 5: $\displaystyle\sum_{k=0}^{5} \binom{5}{k}^3$

\noindent\textcolor{answercolor}{\textbf{ANSWER: (A) $\displaystyle\sum_{k=0}^{5} \binom{5}{k}^3$}}

\vspace{1em}
\hrule
\vspace{1em}

\subsection{SQ18}

\noindent\textcolor{problemcolor}{\textbf{PROBLEM:}}

How many odd positive 3-digit integers are divisible by 3 but do not contain the digit 3?

\begin{enumerate}[label=(\Alph*)]
\item 96
\item 97
\item 98
\item 102
\item 120
\end{enumerate}

\noindent\textcolor{solutioncolor}{\textbf{SOLUTION:}}

Conditions:
\begin{enumerate}
\item 3-digit number: $100 \le n \le 999$
\item Odd: last digit $\in \{1, 5, 7, 9\}$ (not 3 since we exclude 3)
\item Divisible by 3: sum of digits $\equiv 0 \pmod{3}$
\item No digit 3: digits from $\{0, 1, 2, 4, 5, 6, 7, 8, 9\}$
\end{enumerate}

Format: $ABC$ where $A \in \{1,2,4,5,6,7,8,9\}$, $B \in \{0,1,2,4,5,6,7,8,9\}$, $C \in \{1,5,7,9\}$

For $A \in \{1,2,4,5,6,7,8,9\}$:
\begin{itemize}
\item $\equiv 0 \pmod{3}$: $\{6,9\}$ — 2 values
\item $\equiv 1 \pmod{3}$: $\{1,4,7\}$ — 3 values
\item $\equiv 2 \pmod{3}$: $\{2,5,8\}$ — 3 values
\end{itemize}

For $B \in \{0,1,2,4,5,6,7,8,9\}$:
\begin{itemize}
\item $\equiv 0 \pmod{3}$: $\{0,6,9\}$ — 3 values
\item $\equiv 1 \pmod{3}$: $\{1,4,7\}$ — 3 values
\item $\equiv 2 \pmod{3}$: $\{2,5,8\}$ — 3 values
\end{itemize}

For each value of $C$:
\begin{itemize}
\item $C = 1$ (sum $\equiv 1 \bmod 3$): Need $A + B \equiv 2 \pmod{3}$ — 24 combinations
\item $C = 5$ (sum $\equiv 2 \bmod 3$): Need $A + B \equiv 1 \pmod{3}$ — 24 combinations
\item $C = 7$ (sum $\equiv 1 \bmod 3$): Need $A + B \equiv 2 \pmod{3}$ — 24 combinations
\item $C = 9$ (sum $\equiv 0 \bmod 3$): Need $A + B \equiv 0 \pmod{3}$ — 24 combinations
\end{itemize}

Total = $24 + 24 + 24 + 24 = 96$

\noindent\textcolor{answercolor}{\textbf{ANSWER: (A) 96}}

\vspace{1em}
\hrule
\vspace{1em}

\subsection{SQ19}

\noindent\textcolor{problemcolor}{\textbf{PROBLEM:}}

A pair of standard 6-sided fair dice is rolled once. The sum of the numbers rolled determines the diameter of a circle. What is the probability that the numerical value of the area of the circle is less than the numerical value of the circle's circumference?

\begin{enumerate}[label=(\Alph*)]
\item $1/36$
\item $1/12$
\item $1/6$
\item $1/4$
\item $5/18$
\end{enumerate}

\noindent\textcolor{solutioncolor}{\textbf{SOLUTION:}}

Let $d$ be the diameter (sum of dice).

Area = $\pi(d/2)^2 = \pi d^2/4$

Circumference = $\pi d$

We need: $\pi d^2/4 < \pi d$

Simplifying: $d^2/4 < d$

$d^2 < 4d$

$d^2 - 4d < 0$

$d(d - 4) < 0$

Since $d > 0$, we need: $d - 4 < 0$, which means $d < 4$.

So we need the sum to be less than 4: sum $\in \{2, 3\}$

Possible outcomes:
\begin{itemize}
\item Sum = 2: $(1,1)$ — 1 way
\item Sum = 3: $(1,2), (2,1)$ — 2 ways
\end{itemize}

Total favorable outcomes = $1 + 2 = 3$

Total possible outcomes = 36

$P = 3/36 = 1/12$

\noindent\textcolor{answercolor}{\textbf{ANSWER: (B) $1/12$}}

\vspace{1em}
\hrule
\vspace{1em}

\subsection{SQ20}

\noindent\textcolor{problemcolor}{\textbf{PROBLEM:}}

Chlo chooses a real number uniformly at random from the interval $[0, 2017]$. Independently, Laurent chooses a real number uniformly at random from the interval $[0, 4034]$. What is the probability that Laurent's number is greater than Chlo's number?

\begin{enumerate}[label=(\Alph*)]
\item $1/2$
\item $2/3$
\item $3/4$
\item $5/6$
\item $7/8$
\end{enumerate}

\noindent\textcolor{solutioncolor}{\textbf{SOLUTION:}}

Let $C$ be Chlo's number: $C \sim \text{Uniform}[0, 2017]$

Let $L$ be Laurent's number: $L \sim \text{Uniform}[0, 4034]$

We need $P(L > C)$.

For a fixed $C = c \in [0, 2017]$:

$L$ ranges from $c$ to 4034 (favorable region)

Length of favorable interval = $4034 - c$

$P(L > C | C = c) = \dfrac{4034 - c}{4034}$

$P(L > C) = \int_0^{2017} \dfrac{4034 - c}{4034} \times \dfrac{1}{2017} \, dc$

$= \dfrac{1}{2017 \times 4034} \int_0^{2017} (4034 - c) \, dc$

$= \dfrac{1}{2017 \times 4034} \left[4034c - \dfrac{c^2}{2}\right]_0^{2017}$

$= \dfrac{1}{2017 \times 4034} \left[4034 \times 2017 - \dfrac{2017^2}{2}\right]$

$= \dfrac{1}{2017 \times 4034} \times 2017\left[4034 - \dfrac{2017}{2}\right]$

$= \dfrac{1}{4034} \left[4034 - 1008.5\right]$

$= \dfrac{3025.5}{4034} = \dfrac{6051/2}{4034} = \dfrac{6051}{8068} = \dfrac{3}{4}$

\noindent\textcolor{answercolor}{\textbf{ANSWER: (C) $3/4$}}

\vspace{2cm}

\begin{center}
\textit{--- End of Solutions ---}
\end{center}

\end{document}
