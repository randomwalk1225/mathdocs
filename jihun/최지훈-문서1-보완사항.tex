\documentclass[12pt,a4paper]{article}

% Packages
\usepackage{fontspec}
\usepackage{amsmath}
\usepackage{amssymb}
\usepackage{geometry}
\usepackage{xcolor}
\usepackage{enumitem}
\usepackage{booktabs}
\usepackage{array}
\usepackage{tabularx}
\usepackage[unicode]{hyperref}
\usepackage{emoji}

% Font Setup - MaruBuri Korean Font
\setmainfont{MaruBuri}[
    UprightFont=*-Regular,
    BoldFont=*-Bold,
    ItalicFont=*-Light,
    BoldItalicFont=*-SemiBold
]
\setsansfont{MaruBuri}[
    UprightFont=*-Regular,
    BoldFont=*-Bold
]
\newfontfamily\hangulfont{MaruBuri}[
    UprightFont=*-Regular
]

% Page Setup
\geometry{
    left=2.5cm,
    right=2.5cm,
    top=2.5cm,
    bottom=2.5cm
}

% Line Spacing
\linespread{1.5}

% Table Padding
\renewcommand{\arraystretch}{1.8}

\begin{document}

\title{MB240 -- Unit 2 Assignment 검토}
\author{최지훈}
\date{}
\maketitle

문서 \textbf{〈MB240 -- Unit 2 Assignment〉}의 2페이지까지의 내용은 논리적으로 잘 구성되어 있고 수학적·모델링적 설명이 정확합니다. 각 섹션별로 검토 결과를 요약하면 다음과 같습니다.

\section*{\emoji{white-check-mark} Part A: Distance--Time Models}

\subsection*{수학 모델}

\begin{itemize}
\item 두 직선식
\[
\text{Train: } y = -\tfrac{5}{6}x + 50,\quad \text{Bus: } y = -\tfrac{1}{2}x + 50
\]
이 명확히 정의되어 있고, $x$·$y$의 단위도 ``분''과 ``남은 거리(마일)''로 정확히 설정되어 있습니다.

\item \textbf{절편 해석}과 \textbf{도착 시간 계산}(60분, 100분 후) 모두 일관성이 있습니다.

\item 속도 해석:
\begin{itemize}
\item $-5/6$ mile/min → 50 mph
\item $-1/2$ mile/min → 30 mph
\end{itemize}
현실적인 수치이며 타당합니다.

\item \textbf{모델 한계}(정차·대기시간·선형성 한계)도 잘 언급되어 있어, 단순모형으로서의 가정이 명확히 표현되어 있습니다.
\end{itemize}

\subsection*{\emoji{white-check-mark} 결론}

이 부분은 완전하고 수학적·논리적 정확성이 높습니다.
단, 실제 데이터 기반(실제 거리·도착시간) 검증이 추가되면 더 완성도가 높아질 수 있습니다.

\section*{\emoji{white-check-mark} Part B: Coordinates and Distance}

\begin{itemize}
\item 지리좌표를 평면직교좌표로 변환하는 수식:
\[
x = R\cos(\text{latitude})\cos(\text{longitude}), \quad
y = R\cos(\text{latitude})\sin(\text{longitude})
\]
은 구면좌표의 기본 투영 공식으로 타당합니다.

\item Google Maps 거리와 직선거리의 차이에 대한 해석(직선 < 실제 이동거리)도 정확합니다.

\item ``angle from north by arctangent'' 언급은 방향 계산 논리로 적절하지만, 엄밀히는 위·경도 좌표를 변환할 때 경도의 변화에 따라 실제 거리 차이가 latitude에 의존한다는 점을 추가 언급하면 더 완전합니다.
\end{itemize}

\subsection*{\emoji{white-check-mark} 보완 제안}

\begin{itemize}
\item 반지름 $R = 6371\,\text{km}$ 혹은 마일 단위 명시가 있으면 단위 일관성이 명확해집니다.
\item 각도 단위(라디안/도) 명시도 추가하면 수식 재현성이 향상됩니다.
\end{itemize}

\section*{\emoji{white-check-mark} Part C: Investment Equation}

\begin{itemize}
\item 재무식 전개가 일관되고 수치 계산(예: 11.9\% × \$200,000 = \$23,800) 정확합니다.

\item ``수익으로 생활비를 충당 가능하나 안정성 위해 급여 필요''라는 결론은 재무적 해석으로 합리적입니다.

\item 투자식
\[
R(P) = 0.119(1,500,000 - P)
\]
은 $P$에 대한 선형 관계를 명확히 보여줍니다.
\end{itemize}

\subsection*{\emoji{white-check-mark} 보완 제안}

\begin{itemize}
\item ``복리(compound interest)'' 조건이 없는 단리 단위이므로, ``연이율 단리로 가정''이라는 문구 추가 시 해석 명확성 ↑.
\item ``리스크'' 항목(변동성, 세금, 인플레이션 등) 추가하면 응용적 완성도 ↑.
\end{itemize}

\section*{\emoji{warning} 2페이지 이후 검토 관련 안내}

이 파일은 2페이지까지만 완성도가 높고 이후 섹션(추가 분석·문항 풀이 부분)은 초안이거나 불완전할 수 있습니다.
따라서, \textbf{2페이지 이후 부분은 다음 기준으로 검토}할 수 있습니다:

\begin{table}[h]
\centering
\small
\begin{tabular}{|l|l|l|}
\hline
\textbf{점검 항목} & \textbf{평가 기준} & \textbf{조치} \\
\hline
수학식의 논리성 & 기호, 단위, 계산 일관성 & 오류 있으면 수정 \\
모델 해석 & 문장과 그래프/수식 간 대응 & 모호한 문장 구체화 \\
결론의 현실성 & 실제 조건(지리·금융 데이터)과 부합 여부 & 근거 추가 \\
서술 논리 & 가정 → 분석 → 결론 구조 & 누락된 단계 보완 \\
\hline
\end{tabular}
\end{table}

\vspace{1em}

원하시면 2페이지 이후 부분의 텍스트를 자동으로 추출해 ``적절하지 않은 답변 부분''만 하이라이트(예: 잘못된 계산, 논리 불연속, 문장 모순 등)로 표시해드릴 수도 있습니다.
그렇게 해드릴까요?

\end{document}
