\documentclass[12pt,a4paper]{article}
\usepackage{fontspec}
\usepackage{amsmath,amssymb}
\usepackage{enumitem}
\usepackage{geometry}
\usepackage{titlesec}
\usepackage{array}
\usepackage{booktabs}
\usepackage[colorlinks=true,linkcolor=blue,urlcolor=blue]{hyperref}

\geometry{margin=2.5cm}

% Font settings for XeLaTeX
\setmainfont{Times New Roman}
\setsansfont{Arial}

% Title formatting
\titleformat{\section}{\Large\bfseries}{\thesection}{1em}{}
\titleformat{\subsection}{\large\bfseries}{\thesubsection}{1em}{}

% Custom command for answer choices
\newcommand{\choice}[1]{{\upshape (#1)}}

\title{\textbf{TMUA Chapter 1 - Quiz 2:\\Statistics Practices P01}}
\author{}
\date{}

\begin{document}

\maketitle

\noindent\textbf{Time Allowed:} 40 minutes\\
\textbf{Number of Questions:} 15\\
\textbf{Difficulty:} $\star\star\star$

\vspace{1em}
\hrule
\vspace{1em}

\section*{Practice Questions}

\subsection*{Q1}
The mean of $n$ numbers is $p$.\\
The mean of two of these numbers is $q$.\\
The mean of the remaining numbers is 10.\\
Which of the following is a correct expression for $n$ in terms of $p$ and $q$?

\begin{enumerate}[label=\choice{\Alph*}, leftmargin=2cm]
\item $\dfrac{2(q-10)}{(p-10)}$
\item $\dfrac{2(q-10)}{(10-p)}$
\item $\dfrac{2(q-10)}{(p+10)}$
\item $\dfrac{2(10-q)}{(p+10)}$
\item $\dfrac{2(10+q)}{(p-10)}$
\item $\dfrac{2(10+q)}{(10-p)}$
\end{enumerate}

\subsection*{Q2}
The expected number of bottles of water sold in a day at a sports ground is directly proportional to the square of the average outside temperature, in degrees Celsius, for that day.\\
On a day when the average outside temperature is 16°C, 64 bottles of water, the expected number, are sold.\\
On a warmer day, when the average outside temperature is $T$°C, 256 bottles of water are sold, which is 31 bottles more than the expected number for that day.\\
What is the value of $T$?

\begin{enumerate}[label=\choice{\Alph*}, leftmargin=2cm]
\item 7.5
\item $\sqrt{450}$
\item 30
\item 32
\item $\sqrt{1148}$
\item 56.25
\end{enumerate}

\subsection*{Q3}
The mean age of the twenty members of a running club is exactly 28.\\
The mean age increases by exactly 2 years when two new members join.\\
What is the mean age of the two new members?

\begin{enumerate}[label=\choice{\Alph*}, leftmargin=2cm]
\item 20 years
\item 22 years
\item 30 years
\item 40 years
\item 50 years
\item 52 years
\end{enumerate}

\subsection*{Q4}
60\% of a sports club's members are women and the remainder are men.\\
This sports club offers the opportunity to play tennis or cricket. Every member plays exactly one of the two sports.
\begin{itemize}
\item $\frac{2}{5}$ of the male members of the club play cricket
\item $\frac{2}{3}$ of the cricketing members of the club are women
\end{itemize}

What is the probability that a member of the club, chosen at random, is a woman who plays tennis?

\begin{enumerate}[label=\choice{\Alph*}, leftmargin=2cm]
\item $\dfrac{1}{5}$
\item $\dfrac{7}{25}$
\item $\dfrac{1}{3}$
\item $\dfrac{11}{25}$
\item $\dfrac{3}{5}$
\end{enumerate}

\subsection*{Q5}
Pascal, Newton, Galileo and Fermat all took the same test. The average score of all four candidates was 16; Pascal and Newton had an average of 16, Pascal and Fermat had an average of 13, while Newton and Fermat had an average of 18.\\
What was Galileo's score?

\begin{enumerate}[label=\choice{\Alph*}, leftmargin=2cm]
\item 14
\item 15
\item 16
\item 17
\item 18
\end{enumerate}

\subsection*{Q6}
In a survey, people were asked to name their favourite fruit pie. The pie chart shows the outcome. The angles shown are exact with no rounding.\\
What is the smallest number of people who could have been surveyed?

\begin{enumerate}[label=\choice{\Alph*}, leftmargin=2cm]
\item 45
\item 60
\item 80
\item 90
\item 180
\end{enumerate}

\subsection*{Q7}
The table shows statistics relating to the test marks of two groups of students.

\begin{center}
\begin{tabular}{lccc}
\toprule
& \textbf{number of students} & \textbf{mean} & \textbf{range} \\
\midrule
group X & 10 & 36 & 16 \\
group Y & 20 & 48 & 21 \\
\bottomrule
\end{tabular}
\end{center}

The results for the two groups of students are combined.\\
What can be deduced about the mean and range of the combined results?

\begin{enumerate}[label=\choice{\Alph*}, leftmargin=2cm]
\item mean = 40, range $\leq$ 16
\item mean = 40, 16 $<$ range $<$ 21
\item mean = 40, range $\geq$ 21
\item mean = 44, range $\leq$ 16
\item mean = 44, 16 $<$ range $<$ 21
\item mean = 44, range $\geq$ 21
\end{enumerate}

\subsection*{Q8}
A list of $n$ numbers has mean $m$ and a unique mode $d$.\\
Two numbers are removed from the list.\\
The remaining list of numbers also has a unique mode, but this unique mode is not equal to $d$.\\
The mean of the remaining $n - 2$ numbers is $m + 2$.\\
What was the unique mode, $d$, of the original list?

\begin{enumerate}[label=\choice{\Alph*}, leftmargin=2cm]
\item $n - m + 2$
\item $n - m - 2$
\item $n + m - 2$
\item $m + n + 2$
\item $m - n + 2$
\item $m - n - 2$
\end{enumerate}

\subsection*{Q9}
Ms. Blackwell gives an exam to two classes. The mean of the scores of the students in the morning class is 84, and the afternoon class's mean score is 70. The ratio of the number of students in the morning class to the number of students in the afternoon class is $\frac{3}{4}$.\\
What is the mean of the scores of all the students?

\begin{enumerate}[label=\choice{\Alph*}, leftmargin=2cm]
\item 74
\item 75
\item 76
\item 77
\item 78
\end{enumerate}

\subsection*{Q10}
The Dunbar family consists of a mother, a father and some children. The average age of the members of the family is 20, the father is 48 years old, and the average age of the mother and children is 16.\\
How many children are in the family?

\begin{enumerate}[label=\choice{\Alph*}, leftmargin=2cm]
\item 2
\item 3
\item 4
\item 5
\item 6
\end{enumerate}

\subsection*{Q11}
A teacher has a list of marks: 17, 13, 5, 10, 14, 9, 12, 16.\\
Which two marks can be removed without changing the mean?

\begin{enumerate}[label=\choice{\Alph*}, leftmargin=2cm]
\item 12 and 17
\item 5 and 17
\item 9 and 16
\item 10 and 12
\item 10 and 14
\end{enumerate}

\subsection*{Q12}
Jane was playing basketball. After a series of 20 shots, Jane had a success rate of 55\%. Five shots later, her success rate had increased to 56\%.\\
On how many of the last five shots did Jane score?

\begin{enumerate}[label=\choice{\Alph*}, leftmargin=2cm]
\item 1
\item 2
\item 3
\item 4
\item 5
\end{enumerate}

\subsection*{Q13}
Viola has been practising the long jump. At one point, the average distance she had jumped was 3.80 m. Her next jump was 3.99 m and that increased her average to 3.81 m. After the following jump, her average had become 3.82 m.\\
How long was her final jump?

\begin{enumerate}[label=\choice{\Alph*}, leftmargin=2cm]
\item 3.97 m
\item 4.00 m
\item 4.01 m
\item 4.03 m
\item 4.04 m
\end{enumerate}

\subsection*{Q14}
What is the sum of all real numbers $x$ for which the median of the numbers 4, 6, 8, 17 and $x$ is equal to the mean of those five numbers?

\begin{enumerate}[label=\choice{\Alph*}, leftmargin=2cm]
\item $-5$
\item $0$
\item $5$
\item $\dfrac{15}{4}$
\item $\dfrac{35}{4}$
\end{enumerate}

\subsection*{Q15}
In the following list of numbers, the integer $n$ appears $n$ times in the list for $1 \leq n \leq 200$.\\
$1, 2, 2, 3, 3, 3, 4, 4, 4, 4, \ldots, 200, 200, \ldots, 200$

What is the median of the numbers in this list?

\begin{enumerate}[label=\choice{\Alph*}, leftmargin=2cm]
\item 100.5
\item 134
\item 142
\item 150.5
\item 167
\end{enumerate}

\end{document}
