\documentclass[11pt,a4paper]{article}
\usepackage{fontspec}
\usepackage{amsmath,amssymb}
\usepackage{enumitem}
\usepackage{geometry}
\usepackage{titlesec}
\usepackage{xcolor}
\usepackage{booktabs}

\geometry{margin=2cm}

% Font settings
\setmainfont{Times New Roman}
\setsansfont{Arial}

% Color definitions
\definecolor{problemcolor}{RGB}{0,51,102}
\definecolor{solutioncolor}{RGB}{0,102,51}
\definecolor{answercolor}{RGB}{153,0,0}

\title{\textbf{TMUA Chapter 1 - Quiz 2:\\Statistics Practices P01\\(With Solutions)}}
\author{Xie Tao TMUA Workbook 2024 (5th Edition)}
\date{}

\begin{document}

\maketitle

\noindent\textbf{Time Allowed:} 40 minutes\\
\textbf{Number of Questions:} 15\\
\textbf{Difficulty:} $\star\star\star$

\vspace{1em}
\hrule
\vspace{1em}

\section{Practice Questions}

\subsection{Q1}

\noindent\textcolor{problemcolor}{\textbf{PROBLEM:}}

The mean of $n$ numbers is $p$.
The mean of two of these numbers is $q$.
The mean of the remaining numbers is 10.
Which of the following is a correct expression for $n$ in terms of $p$ and $q$?

\begin{enumerate}[label=(\Alph*)]
\item $2(q - 10) / (p - 10)$
\item $2(q - 10) / (10 - p)$
\item $2(q - 10) / (p + 10)$
\item $2(10 - q) / (p + 10)$
\item $2(10 + q) / (p - 10)$
\item $2(10 + q) / (10 - p)$
\end{enumerate}

\noindent\textcolor{solutioncolor}{\textbf{SOLUTION:}}

Total sum of $n$ numbers = $np$

Sum of two numbers = $2q$

Sum of remaining $(n - 2)$ numbers = $10(n - 2)$

Setting up the equation:

$np = 2q + 10(n - 2)$

$np = 2q + 10n - 20$

$np - 10n = 2q - 20$

$n(p - 10) = 2(q - 10)$

$n = \dfrac{2(q - 10)}{p - 10}$

\noindent\textcolor{answercolor}{\textbf{ANSWER: (A) $2(q - 10) / (p - 10)$}}

\vspace{1em}
\hrule
\vspace{1em}

\subsection{Q2}

\noindent\textcolor{problemcolor}{\textbf{PROBLEM:}}

The expected number of bottles of water sold in a day at a sports ground is directly proportional to the square of the average outside temperature, in degrees Celsius, for that day.
On a day when the average outside temperature is 16°C, 64 bottles of water, the expected number, are sold.
On a warmer day, when the average outside temperature is $T$°C, 256 bottles of water are sold, which is 31 bottles more than the expected number for that day.
What is the value of $T$?

\begin{enumerate}[label=(\Alph*)]
\item 7.5
\item $\sqrt{450}$
\item 30
\item 32
\item $\sqrt{1148}$
\item 56.25
\end{enumerate}

\noindent\textcolor{solutioncolor}{\textbf{SOLUTION:}}

Let bottles = $k \times \text{(temperature)}^2$

From the first day: $64 = k \times 16^2$

$64 = k \times 256$

$k = \dfrac{64}{256} = \dfrac{1}{4}$

On the warmer day:
\begin{itemize}
\item Actual bottles sold = 256
\item Expected bottles = $256 - 31 = 225$
\end{itemize}

Expected bottles = $k \times T^2$

$225 = \dfrac{1}{4} \times T^2$

$T^2 = 900$

$T = 30$

\noindent\textcolor{answercolor}{\textbf{ANSWER: (C) 30}}

\vspace{1em}
\hrule
\vspace{1em}

\subsection{Q3}

\noindent\textcolor{problemcolor}{\textbf{PROBLEM:}}

The mean age of the twenty members of a running club is exactly 28.
The mean age increases by exactly 2 years when two new members join.
What is the mean age of the two new members?

\begin{enumerate}[label=(\Alph*)]
\item 20 years
\item 22 years
\item 30 years
\item 40 years
\item 50 years
\item 52 years
\end{enumerate}

\noindent\textcolor{solutioncolor}{\textbf{SOLUTION:}}

Total age of 20 members = $20 \times 28 = 560$ years

New mean with 22 members = $28 + 2 = 30$ years

Total age with 22 members = $22 \times 30 = 660$ years

Age of two new members = $660 - 560 = 100$ years

Mean age of two new members = $100 / 2 = 50$ years

\noindent\textcolor{answercolor}{\textbf{ANSWER: (E) 50 years}}

\vspace{1em}
\hrule
\vspace{1em}

\subsection{Q4}

\noindent\textcolor{problemcolor}{\textbf{PROBLEM:}}

60\% of a sports club's members are women and the remainder are men.
This sports club offers the opportunity to play tennis or cricket. Every member plays exactly one of the two sports.
\begin{itemize}
\item $\frac{2}{5}$ of the male members of the club play cricket
\item $\frac{2}{3}$ of the cricketing members of the club are women
\end{itemize}

What is the probability that a member of the club, chosen at random, is a woman who plays tennis?

\begin{enumerate}[label=(\Alph*)]
\item $\frac{1}{5}$
\item $\frac{7}{25}$
\item $\frac{1}{3}$
\item $\frac{11}{25}$
\item $\frac{3}{5}$
\end{enumerate}

\noindent\textcolor{solutioncolor}{\textbf{SOLUTION:}}

Assume 100 members total.

Women = 60, Men = 40

Male cricket players = $\frac{2}{5} \times 40 = 16$

If $\frac{2}{3}$ of cricket players are women, then $\frac{1}{3}$ are men:

$\frac{1}{3} \times (\text{total cricket players}) = 16$

Total cricket players = 48

Female cricket players = $\frac{2}{3} \times 48 = 32$

Female tennis players = $60 - 32 = 28$

Probability = $\frac{28}{100} = \frac{7}{25}$

\noindent\textcolor{answercolor}{\textbf{ANSWER: (B) $\frac{7}{25}$}}

\vspace{1em}
\hrule
\vspace{1em}

\subsection{Q5}

\noindent\textcolor{problemcolor}{\textbf{PROBLEM:}}

Pascal, Newton, Galileo and Fermat all took the same test. The average score of all four candidates was 16; Pascal and Newton had an average of 16, Pascal and Fermat had an average of 13, while Newton and Fermat had an average of 18.
What was Galileo's score?

\begin{enumerate}[label=(\Alph*)]
\item 14
\item 15
\item 16
\item 17
\item 18
\end{enumerate}

\noindent\textcolor{solutioncolor}{\textbf{SOLUTION:}}

Let P = Pascal, N = Newton, G = Galileo, F = Fermat

$P + N + G + F = 4 \times 16 = 64$ \ldots(1)

$P + N = 2 \times 16 = 32$ \ldots(2)

$P + F = 2 \times 13 = 26$ \ldots(3)

$N + F = 2 \times 18 = 36$ \ldots(4)

From (2): $G + F = 64 - 32 = 32$ \ldots(5)

From (3): $P = 26 - F$

From (4): $N = 36 - F$

Substituting into (2):

$(26 - F) + (36 - F) = 32$

$62 - 2F = 32$

$2F = 30$

$F = 15$

From (5): $G = 32 - 15 = 17$

\noindent\textcolor{answercolor}{\textbf{ANSWER: (D) 17}}

\vspace{1em}
\hrule
\vspace{1em}

\subsection{Q6}

\noindent\textcolor{problemcolor}{\textbf{PROBLEM:}}

In a survey, people were asked to name their favourite fruit pie. The pie chart shows the outcome. The angles shown are exact with no rounding.
What is the smallest number of people who could have been surveyed?

\begin{enumerate}[label=(\Alph*)]
\item 45
\item 60
\item 80
\item 90
\item 180
\end{enumerate}

\noindent\textcolor{solutioncolor}{\textbf{SOLUTION:}}

For the angles to be exact (no rounding), the number of people for each category must be a whole number.

Each person represents $\frac{360^\circ}{n}$ of the pie chart, where $n$ is the total number of people.

For all angles to be exact multiples of $\frac{360^\circ}{n}$, $n$ must be chosen such that all resulting angles are integers.

Typical pie chart angles are multiples of common factors. The most restrictive common requirement is that $n$ divides 360 evenly with appropriate fractions.

Testing options:
\begin{itemize}
\item For angles like 120°, 90°, 60°, 45°, 30°, etc., we need $n$ such that all fractions work out to whole numbers
\item The LCM consideration suggests $n = 60$ or 180
\end{itemize}

For minimum people with exact angles (no decimals), the answer is typically a factor of 360 that allows common divisions.

\noindent\textcolor{answercolor}{\textbf{ANSWER: (B) 60}}

\vspace{1em}
\hrule
\vspace{1em}

\subsection{Q7}

\noindent\textcolor{problemcolor}{\textbf{PROBLEM:}}

The table shows statistics relating to the test marks of two groups of students.

\begin{center}
\begin{tabular}{lccc}
\toprule
& number of students & mean & range \\
\midrule
group X & 10 & 36 & 16 \\
group Y & 20 & 48 & 21 \\
\bottomrule
\end{tabular}
\end{center}

The results for the two groups of students are combined.
What can be deduced about the mean and range of the combined results?

\begin{enumerate}[label=(\Alph*)]
\item mean = 40, range $\leq$ 16
\item mean = 40, 16 < range < 21
\item mean = 40, range $\geq$ 21
\item mean = 44, range $\leq$ 16
\item mean = 44, 16 < range < 21
\item mean = 44, range $\geq$ 21
\end{enumerate}

\noindent\textcolor{solutioncolor}{\textbf{SOLUTION:}}

\textbf{Combined mean:}

Total sum = $10 \times 36 + 20 \times 48 = 360 + 960 = 1320$

Combined mean = $1320 / 30 = 44$

\textbf{Combined range:}

Group X range = 16 means scores span at least 16 points

Group Y range = 21 means scores span at least 21 points

The combined range is at least as large as the larger individual range.

Therefore: combined range $\geq$ 21

\noindent\textcolor{answercolor}{\textbf{ANSWER: (F) mean = 44, range $\geq$ 21}}

\vspace{1em}
\hrule
\vspace{1em}

\subsection{Q8}

\noindent\textcolor{problemcolor}{\textbf{PROBLEM:}}

A list of $n$ numbers has mean $m$ and a unique mode $d$.
Two numbers are removed from the list.
The remaining list of numbers also has a unique mode, but this unique mode is not equal to $d$.
The mean of the remaining $n - 2$ numbers is $m + 2$.
What was the unique mode, $d$, of the original list?

\begin{enumerate}[label=(\Alph*)]
\item $n - m + 2$
\item $n - m - 2$
\item $n + m - 2$
\item $m + n + 2$
\item $m - n + 2$
\item $m - n - 2$
\end{enumerate}

\noindent\textcolor{solutioncolor}{\textbf{SOLUTION:}}

Sum of original $n$ numbers = $nm$

Sum of remaining $(n - 2)$ numbers = $(n - 2)(m + 2) = nm - 2m + 2n - 4$

Sum of two removed numbers = $nm - (nm - 2m + 2n - 4)$

$= 2m - 2n + 4$

$= 2(m - n + 2)$

Since the mode changed when two numbers were removed, both removed numbers must have been $d$ (the original mode).

Therefore: $2d = 2(m - n + 2)$

$d = m - n + 2$

\noindent\textcolor{answercolor}{\textbf{ANSWER: (E) $m - n + 2$}}

\vspace{1em}
\hrule
\vspace{1em}

\subsection{Q9}

\noindent\textcolor{problemcolor}{\textbf{PROBLEM:}}

Ms. Blackwell gives an exam to two classes. The mean of the scores of the students in the morning class is 84, and the afternoon class's mean score is 70. The ratio of the number of students in the morning class to the number of students in the afternoon class is $\frac{3}{4}$.
What is the mean of the scores of all the students?

\begin{enumerate}[label=(\Alph*)]
\item 74
\item 75
\item 76
\item 77
\item 78
\end{enumerate}

\noindent\textcolor{solutioncolor}{\textbf{SOLUTION:}}

Let morning class have $3k$ students, afternoon class have $4k$ students.

Total sum = $3k \times 84 + 4k \times 70$

$= 252k + 280k$

$= 532k$

Total students = $3k + 4k = 7k$

Mean = $\dfrac{532k}{7k} = \dfrac{532}{7} = 76$

\noindent\textcolor{answercolor}{\textbf{ANSWER: (C) 76}}

\vspace{1em}
\hrule
\vspace{1em}

\subsection{Q10}

\noindent\textcolor{problemcolor}{\textbf{PROBLEM:}}

The Dunbar family consists of a mother, a father and some children. The average age of the members of the family is 20, the father is 48 years old, and the average age of the mother and children is 16.
How many children are in the family?

\begin{enumerate}[label=(\Alph*)]
\item 2
\item 3
\item 4
\item 5
\item 6
\end{enumerate}

\noindent\textcolor{solutioncolor}{\textbf{SOLUTION:}}

Let $c$ = number of children

Total family members = $c + 2$ (mother + father + children)

Total age of family = $20(c + 2) = 20c + 40$

Father's age = 48

Mother and children's total age = $20c + 40 - 48 = 20c - 8$

Average age of mother and children = 16

$\dfrac{20c - 8}{c + 1} = 16$

$20c - 8 = 16c + 16$

$4c = 24$

$c = 6$

\noindent\textcolor{answercolor}{\textbf{ANSWER: (E) 6}}

\vspace{1em}
\hrule
\vspace{1em}

\subsection{Q11}

\noindent\textcolor{problemcolor}{\textbf{PROBLEM:}}

A teacher has a list of marks: 17, 13, 5, 10, 14, 9, 12, 16.
Which two marks can be removed without changing the mean?

\begin{enumerate}[label=(\Alph*)]
\item 12 and 17
\item 5 and 17
\item 9 and 16
\item 10 and 12
\item 10 and 14
\end{enumerate}

\noindent\textcolor{solutioncolor}{\textbf{SOLUTION:}}

Sum of all marks = $17 + 13 + 5 + 10 + 14 + 9 + 12 + 16 = 96$

Mean = $96 / 8 = 12$

For the mean to remain 12 after removing two marks, those two marks must also have a mean of 12 (sum = 24).

Testing options:
\begin{itemize}
\item (A) $12 + 17 = 29$
\item (B) $5 + 17 = 22$
\item (C) $9 + 16 = 25$
\item (D) $10 + 12 = 22$
\item (E) $10 + 14 = 24$ \checkmark
\end{itemize}

\noindent\textcolor{answercolor}{\textbf{ANSWER: (E) 10 and 14}}

\vspace{1em}
\hrule
\vspace{1em}

\subsection{Q12}

\noindent\textcolor{problemcolor}{\textbf{PROBLEM:}}

Jane was playing basketball. After a series of 20 shots, Jane had a success rate of 55\%. Five shots later, her success rate had increased to 56\%.
On how many of the last five shots did Jane score?

\begin{enumerate}[label=(\Alph*)]
\item 1
\item 2
\item 3
\item 4
\item 5
\end{enumerate}

\noindent\textcolor{solutioncolor}{\textbf{SOLUTION:}}

After 20 shots: successes = $0.55 \times 20 = 11$

After 25 shots: successes = $0.56 \times 25 = 14$

Successes in last 5 shots = $14 - 11 = 3$

\noindent\textcolor{answercolor}{\textbf{ANSWER: (C) 3}}

\vspace{1em}
\hrule
\vspace{1em}

\subsection{Q13}

\noindent\textcolor{problemcolor}{\textbf{PROBLEM:}}

Viola has been practising the long jump. At one point, the average distance she had jumped was 3.80 m. Her next jump was 3.99 m and that increased her average to 3.81 m. After the following jump, her average had become 3.82 m.
How long was her final jump?

\begin{enumerate}[label=(\Alph*)]
\item 3.97 m
\item 4.00 m
\item 4.01 m
\item 4.03 m
\item 4.04 m
\end{enumerate}

\noindent\textcolor{solutioncolor}{\textbf{SOLUTION:}}

Let $n$ = number of jumps initially

Total distance initially = $3.80n$

After jump of 3.99 m:

$\dfrac{3.80n + 3.99}{n + 1} = 3.81$

$3.80n + 3.99 = 3.81n + 3.81$

$0.18 = 0.01n$

$n = 18$

After 19 jumps: total = $3.81 \times 19 = 72.39$ m

Let final jump = $x$

$\dfrac{72.39 + x}{20} = 3.82$

$72.39 + x = 76.40$

$x = 4.01$ m

\noindent\textcolor{answercolor}{\textbf{ANSWER: (C) 4.01 m}}

\vspace{1em}
\hrule
\vspace{1em}

\subsection{Q14}

\noindent\textcolor{problemcolor}{\textbf{PROBLEM:}}

What is the sum of all real numbers $x$ for which the median of the numbers 4, 6, 8, 17 and $x$ is equal to the mean of those five numbers?

\begin{enumerate}[label=(\Alph*)]
\item $-5$
\item 0
\item 5
\item $\frac{15}{4}$
\item $\frac{35}{4}$
\end{enumerate}

\noindent\textcolor{solutioncolor}{\textbf{SOLUTION:}}

Sum of known numbers = $4 + 6 + 8 + 17 = 35$

Mean = $\dfrac{35 + x}{5}$

\textbf{Case 1:} $x \leq 4$

Sorted: $x, 4, 6, 8, 17$

Median = 6

$6 = \dfrac{35 + x}{5}$

$30 = 35 + x$

$x = -5$ \checkmark (satisfies $x \leq 4$)

\textbf{Case 2:} $4 < x \leq 6$

Sorted: $4, x, 6, 8, 17$

Median = 6

$6 = \dfrac{35 + x}{5}$

$x = -5$ (doesn't satisfy $4 < x \leq 6$)

\textbf{Case 3:} $6 < x \leq 8$

Sorted: $4, 6, x, 8, 17$

Median = $x$

$x = \dfrac{35 + x}{5}$

$5x = 35 + x$

$4x = 35$

$x = \dfrac{35}{4} = 8.75$ (doesn't satisfy $x \leq 8$)

\textbf{Case 4:} $8 < x \leq 17$

Sorted: $4, 6, 8, x, 17$

Median = 8

$8 = \dfrac{35 + x}{5}$

$40 = 35 + x$

$x = 5$ (doesn't satisfy $8 < x$)

\textbf{Case 5:} $x > 17$

Sorted: $4, 6, 8, 17, x$

Median = 8

$8 = \dfrac{35 + x}{5}$

$x = 5$ (doesn't satisfy $x > 17$)

Sum of all valid $x$ values = $-5$

\noindent\textcolor{answercolor}{\textbf{ANSWER: (A) $-5$}}

\vspace{1em}
\hrule
\vspace{1em}

\subsection{Q15}

\noindent\textcolor{problemcolor}{\textbf{PROBLEM:}}

In the following list of numbers, the integer $n$ appears $n$ times in the list for $1 \leq n \leq 200$.

1, 2, 2, 3, 3, 3, 4, 4, 4, 4, \ldots, 200, 200, \ldots, 200

What is the median of the numbers in this list?

\begin{enumerate}[label=(\Alph*)]
\item 100.5
\item 134
\item 142
\item 150.5
\item 167
\end{enumerate}

\noindent\textcolor{solutioncolor}{\textbf{SOLUTION:}}

Total count of numbers = $1 + 2 + 3 + \ldots + 200 = \dfrac{200 \times 201}{2} = 20100$

Median position = $\dfrac{20100 + 1}{2} = 10050.5$

So we need the average of the 10050th and 10051st numbers.

We need to find which number $n$ contains position 10050.

Numbers 1 through $n$ account for $1 + 2 + 3 + \ldots + n = \dfrac{n(n + 1)}{2}$ positions.

We need $\dfrac{n(n + 1)}{2} \approx 10050$

$n^2 + n \approx 20100$

$n \approx \sqrt{20100} \approx 141.8$

Testing $n = 141$: $\dfrac{141 \times 142}{2} = 10011$

Testing $n = 142$: $\dfrac{142 \times 143}{2} = 10153$

So positions 10012 through 10153 all contain the number 142.

Position 10050 falls in this range.

Position 10051 also falls in this range.

Median = $\dfrac{142 + 142}{2} = 142$

\noindent\textcolor{answercolor}{\textbf{ANSWER: (C) 142}}

\vspace{2cm}

\begin{center}
\textit{--- End of Solutions ---}
\end{center}

\end{document}
