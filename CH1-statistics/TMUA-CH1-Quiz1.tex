\documentclass[12pt,a4paper]{article}
\usepackage{fontspec}
\usepackage{amsmath,amssymb}
\usepackage{enumitem}
\usepackage{geometry}
\usepackage{titlesec}
\usepackage{array}
\usepackage{booktabs}
\usepackage[colorlinks=true,linkcolor=blue,urlcolor=blue]{hyperref}

\geometry{margin=2.5cm}

% Font settings for XeLaTeX
\setmainfont{Times New Roman}
\setsansfont{Arial}

% Title formatting
\titleformat{\section}{\Large\bfseries}{\thesection}{1em}{}
\titleformat{\subsection}{\large\bfseries}{\thesubsection}{1em}{}

% Custom command for answer choices
\newcommand{\choice}[1]{{\upshape (#1)}}

\title{\textbf{TMUA Chapter 1 - Quiz 1:\\Statistics Exercises E01}}
\author{Xie Tao TMUA Workbook 2024 (5th Edition)}
\date{}

\begin{document}

\maketitle

\noindent\textbf{Time Allowed:} No limit\\
\textbf{Number of Questions:} 23\\
\textbf{Difficulty:} $\star\star\star\circ$

\vspace{1em}
\hrule
\vspace{1em}

\section*{Pre-Quiz Questions}

\subsection*{Quiz Pre-1}
Four different positive integers are to be chosen so that they have a mean of 2017.\\
What is the smallest possible range of the chosen integers?

\begin{enumerate}[label=\choice{\Alph*}, leftmargin=2cm]
\item 2
\item 3
\item 4
\item 5
\item 6
\end{enumerate}

\subsection*{Quiz Pre-2}
Frank's teacher asks him to write down five integers such that the median is one more than the mean, and the mode is one greater than the median. Frank is also told that the median is 10.\\
What is the smallest possible integer that he could include in his list?

\begin{enumerate}[label=\choice{\Alph*}, leftmargin=2cm]
\item 3
\item 4
\item 5
\item 6
\item 7
\end{enumerate}

\subsection*{Quiz Pre-3}
The mean, median and mode of the 7 data values 60, 100, $x$, 40, 50, 200, 90 are all equal to $x$.\\
What is the value of $x$?

\begin{enumerate}[label=\choice{\Alph*}, leftmargin=2cm]
\item 50
\item 60
\item 75
\item 90
\item 100
\end{enumerate}

\vspace{1em}
\hrule
\vspace{1em}

\section*{Exercise Questions}

\subsection*{Ex. 1}
The data set [6, 19, 33, 33, 39, 41, 41, 43, 51, 57] has median $Q_2 = 40$, first quartile $Q_1 = 33$, and third quartile $Q_3 = 43$. An outlier in a data set is a value that is more than 1.5 times the interquartile range below the first quartile ($Q_1$) or more than 1.5 times the interquartile range above the third quartile ($Q_3$), where the interquartile range is defined as $Q_3 - Q_1$.\\
How many outliers does this data set have?

\begin{enumerate}[label=\choice{\Alph*}, leftmargin=2cm]
\item 0
\item 1
\item 2
\item 3
\item 4
\end{enumerate}

\subsection*{Ex. 2}
A list of five numbers has mean $x$, median $y$ and range $z$.\\
A sixth number is added to the list. This sixth number is greater than $x$.\\
Which of the following statements must be true?

\begin{enumerate}[label=\arabic*., leftmargin=2cm]
\item The median of the six numbers cannot be one of the numbers in the list.
\item The mean of the six numbers is greater than $x$.
\item The range of the six numbers is greater than $z$.
\end{enumerate}

\begin{enumerate}[label=\choice{\Alph*}, leftmargin=2cm]
\item none of them
\item 1 only
\item 2 only
\item 3 only
\item 1 and 2 only
\item 1 and 3 only
\item 2 and 3 only
\item 1, 2 and 3
\end{enumerate}

\subsection*{Ex. 3}
There are two sets of data: the mean of the first set is 15, and the mean of the second set is 20.\\
One of the pieces of data from the first set is exchanged with one of the pieces of data from the second set.\\
As a result, the mean of the first set of data increases from 15 to 16, and the mean of the second set of data decreases from 20 to 17.\\
What is the mean of the set made by combining all the data?

\begin{enumerate}[label=\choice{\Alph*}, leftmargin=2cm]
\item $16\frac{1}{4}$
\item $16\frac{1}{3}$
\item $16\frac{1}{2}$
\item $16\frac{2}{3}$
\item $16\frac{3}{4}$
\end{enumerate}

\subsection*{Ex. 4}
A class of 20 students took a maths test, and their mean mark was 70. The range of these marks was 18.\\
Five new students joined the class and took the same maths test. When their marks were included, the new mean for the 25 students was 68.\\
Given only this information, which of the following statements must be true?

\begin{enumerate}[label=\arabic*., leftmargin=2cm]
\item All of the five new students scored 68 marks or less for this test.
\item The mean of the marks for just the five new students was 60.
\item When the marks for the five new students were included, the range of the marks for the class was unchanged.
\end{enumerate}

\begin{enumerate}[label=\choice{\Alph*}, leftmargin=2cm]
\item none of them
\item 1 only
\item 2 only
\item 3 only
\item 1 and 2 only
\item 1 and 3 only
\item 2 and 3 only
\item 1, 2 and 3
\end{enumerate}

\subsection*{Ex. 5}
A group of five numbers are such that:
\begin{itemize}
\item their mean is 0
\item their range is 20
\end{itemize}

What is the largest possible median of the five numbers?

\begin{enumerate}[label=\choice{\Alph*}, leftmargin=2cm]
\item 0
\item 4
\item $4\frac{1}{2}$
\item $6\frac{1}{2}$
\item 8
\item 20
\end{enumerate}

\subsection*{Ex. 6}
Melanie computes the mean $\mu$, the median $M$ and the modes of the 365 values that are the dates in the months of 2019. Thus her data consist of 12 1s, 12 2s, \ldots, 12 28s, 11 29s, 11 30s and 7 31s. Let $d$ be the median of the modes.\\
Which of the following statements is true?

\begin{enumerate}[label=\choice{\Alph*}, leftmargin=2cm]
\item $\mu < d < M$
\item $M < d < \mu$
\item $d = M = \mu$
\item $d < M < \mu$
\item $d < \mu < M$
\end{enumerate}

\subsection*{Ex. 7}
A list of 2018 positive integers has a unique mode, which occurs exactly 10 times.\\
What is the least number of distinct values that can occur in the list?

\begin{enumerate}[label=\choice{\Alph*}, leftmargin=2cm]
\item 202
\item 223
\item 224
\item 225
\item 234
\end{enumerate}

\subsection*{Ex. 8}
What is the median of the following list of 4040 numbers?\\
$1, 2, 3, \ldots, 2020, 1^2, 2^2, 3^2, \ldots, 2020^2$

\begin{enumerate}[label=\choice{\Alph*}, leftmargin=2cm]
\item 1974.5
\item 1975.5
\item 1976.5
\item 1977.5
\item 1978.5
\end{enumerate}

\subsection*{Ex. 9}
The mean, median, unique mode and range of a collection of eight integers are all equal to 8.\\
The largest integer that can be an element of this collection is

\begin{enumerate}[label=\choice{\Alph*}, leftmargin=2cm]
\item 11
\item 12
\item 13
\item 14
\item 15
\end{enumerate}

\subsection*{Ex. 10}
A list of 11 positive integers has a mean of 10, a median of 9 and a unique mode of 8.\\
What is the largest possible value of an integer in the list?

\begin{enumerate}[label=\choice{\Alph*}, leftmargin=2cm]
\item 24
\item 30
\item 31
\item 33
\item 35
\end{enumerate}

\subsection*{Ex. 11}
When 15 is appended to a list of integers, the mean is increased by 2. When 1 is appended to the enlarged list, the mean of the enlarged list is decreased by 1.\\
How many integers were in the original list?

\begin{enumerate}[label=\choice{\Alph*}, leftmargin=2cm]
\item 4
\item 5
\item 6
\item 7
\item 8
\end{enumerate}

\subsection*{Ex. 12}
Every high school in the city of Euclid sent a team of 3 students to a math contest. Each participant in the contest received a different score. Andrea's score was the median among all students, and hers was the highest score on her team. Andrea's teammates Beth and Carla placed 37th and 64th, respectively.\\
How many schools are in the city?

\begin{enumerate}[label=\choice{\Alph*}, leftmargin=2cm]
\item 22
\item 23
\item 24
\item 25
\item 26
\end{enumerate}

\subsection*{Ex. 13}
On a certain math exam, 10\% of the students got 70 points, 25\% got 80 points, 20\% got 85 points, 15\% got 90 points and the rest got 95 points.\\
What is the difference between the mean and the median score on this exam?

\begin{enumerate}[label=\choice{\Alph*}, leftmargin=2cm]
\item 0
\item 1
\item 2
\item 4
\item 5
\end{enumerate}

\subsection*{Ex. 14}
Johann has 64 fair coins. He flips all the coins. Any coin that lands on tails is tossed again. Coins that land on tails on the second toss are tossed a third time.\\
What is the expected number of coins that are now heads?

\begin{enumerate}[label=\choice{\Alph*}, leftmargin=2cm]
\item 32
\item 40
\item 48
\item 56
\item 64
\end{enumerate}

\vspace{1em}
\hrule
\vspace{1em}

\section*{Quiz Questions}

\subsection*{Quiz 1}
The following twelve integers are written in ascending order:\\
$1, x, x, x, y, y, y, y, y, 8, 9, 11$.

The mean of these twelve integers is 7.\\
What is the median?

\begin{enumerate}[label=\choice{\Alph*}, leftmargin=2cm]
\item 6
\item 7
\item 7.5
\item 8
\item 9
\end{enumerate}

\subsection*{Quiz 2}
A list of positive integers has a median of 8, a mode of 9 and a mean of 10.\\
What is the smallest possible number of integers in the list?

\begin{enumerate}[label=\choice{\Alph*}, leftmargin=2cm]
\item 5
\item 6
\item 7
\item 8
\item 9
\end{enumerate}

\subsection*{Quiz 3}
What is the maximum possible value of the median number of cups of coffee bought per customer on a day when Sundollars Coffee Shop sells 477 cups of coffee to 190 customers, and every customer buys at least one cup of coffee?

\begin{enumerate}[label=\choice{\Alph*}, leftmargin=2cm]
\item 1.5
\item 2
\item 2.5
\item 3
\item 3.5
\end{enumerate}

\subsection*{Quiz 4}
A list of 5 positive integers has mean 5, mode 5, median 5 and range 5.\\
How many such lists of 5 positive integers are there?

\begin{enumerate}[label=\choice{\Alph*}, leftmargin=2cm]
\item 1
\item 2
\item 3
\item 4
\item 5
\end{enumerate}

\subsection*{Quiz 5}
A set of six distinct integers is split into two sets of three.\\
The first set of three integers has a mean of 10 and a median of 8.\\
The second set of three integers has a mean of 12 and a median of 9.\\
What is the smallest possible range of the set of all six integers?

\begin{enumerate}[label=\choice{\Alph*}, leftmargin=2cm]
\item 8
\item 10
\item 11
\item 12
\item 14
\item 15
\end{enumerate}

\subsection*{Quiz 6}
The table shows statistics relating to the test marks of two groups of students.

\begin{center}
\begin{tabular}{lccc}
\toprule
& \textbf{number of students} & \textbf{mean} & \textbf{range} \\
\midrule
group X & 10 & 36 & 16 \\
group Y & 20 & 48 & 21 \\
\bottomrule
\end{tabular}
\end{center}

The results for the two groups of students are combined.\\
What can be deduced about the mean and range of the combined results?

\begin{enumerate}[label=\choice{\Alph*}, leftmargin=2cm]
\item mean = 40, range $\leq$ 16
\item mean = 40, 16 $<$ range $<$ 21
\item mean = 40, range $\geq$ 21
\item mean = 44, range $\leq$ 16
\item mean = 44, 16 $<$ range $<$ 21
\item mean = 44, range $\geq$ 21
\end{enumerate}

\end{document}
