\documentclass[12pt,a4paper]{article}
\usepackage{fontspec}
\usepackage{amsmath,amssymb}
\usepackage{enumitem}
\usepackage{geometry}
\usepackage{titlesec}
\usepackage{array}
\usepackage{booktabs}
\usepackage[colorlinks=true,linkcolor=blue,urlcolor=blue]{hyperref}

\geometry{margin=2.5cm}

% Font settings for XeLaTeX
\setmainfont{Times New Roman}
\setsansfont{Arial}

% Title formatting
\titleformat{\section}{\Large\bfseries}{\thesection}{1em}{}
\titleformat{\subsection}{\large\bfseries}{\thesubsection}{1em}{}

% Custom command for answer choices
\newcommand{\choice}[1]{{\upshape (#1)}}

\title{\textbf{TMUA Chapter 1 - Quiz 3:\\Statistics Supplements S01}}
\author{}
\date{}

\begin{document}

\maketitle

\noindent\textbf{Time Allowed:} 90 minutes\\
\textbf{Number of Questions:} 15\\
\textbf{Difficulty:} $\star\star\star\star$

\vspace{1em}
\hrule
\vspace{1em}

\section*{Supplement Questions}

\subsection*{SQ1}
Five positive consecutive integers starting with $a$ have average $b$.\\
What is the average of 5 consecutive integers that start with $b$?

\begin{enumerate}[label=\choice{\Alph*}, leftmargin=2cm]
\item $a + 3$
\item $a + 4$
\item $a + 5$
\item $a + 6$
\item $a + 7$
\end{enumerate}

\subsection*{SQ2}
An iterative average of the numbers 1, 2, 3, 4 and 5 is computed in the following way. Arrange the five numbers in some order. Find the mean of the first two numbers, then find the mean of that with the third number, then the mean of that with the fourth number, and finally the mean of that with the fifth number.\\
What is the difference between the largest and smallest possible values that can be obtained using this procedure?

\begin{enumerate}[label=\choice{\Alph*}, leftmargin=2cm]
\item $\dfrac{31}{16}$
\item $2$
\item $\dfrac{17}{8}$
\item $3$
\item $\dfrac{65}{16}$
\end{enumerate}

\subsection*{SQ3}
Mrs. Walter gave an exam in a mathematics class of five students. She entered the scores in random order into a spreadsheet, which recalculated the class average after each score was entered. Mrs. Walter noticed that after each score was entered, the average was always an integer. The scores (listed in ascending order) were 71, 76, 80, 82 and 91.\\
What was the last score Mrs. Walter entered?

\begin{enumerate}[label=\choice{\Alph*}, leftmargin=2cm]
\item 71
\item 76
\item 80
\item 82
\item 91
\end{enumerate}

\subsection*{SQ4}
The average value of all the pennies, nickels, dimes and quarters in Paula's purse is 20 cents. If she had one more quarter, the average value would be 21 cents.\\
How many dimes does she have in her purse?

\begin{enumerate}[label=\choice{\Alph*}, leftmargin=2cm]
\item 0
\item 1
\item 2
\item 3
\item 4
\end{enumerate}

\subsection*{SQ5}
The average of the numbers 1, 2, 3, \ldots, 98, 99 and $x$ is $100x$.\\
What is $x$?

\begin{enumerate}[label=\choice{\Alph*}, leftmargin=2cm]
\item $\dfrac{49}{101}$
\item $\dfrac{50}{101}$
\item $\dfrac{1}{2}$
\item $\dfrac{51}{101}$
\item $\dfrac{50}{99}$
\end{enumerate}

\subsection*{SQ6}
The mean of three numbers is 10 more than the least of the numbers and 15 less than the greatest. The median of the three numbers is 5.\\
What is their sum?

\begin{enumerate}[label=\choice{\Alph*}, leftmargin=2cm]
\item 5
\item 20
\item 25
\item 30
\item 36
\end{enumerate}

\subsection*{SQ7}
A teacher gave a test to a class in which 10\% of the students are juniors and 90\% are seniors. The average score on the test was 84. The juniors all received the same score, and the average score of the seniors was 83.\\
What score did each of the juniors receive on the test?

\begin{enumerate}[label=\choice{\Alph*}, leftmargin=2cm]
\item 85
\item 88
\item 93
\item 94
\item 98
\end{enumerate}

\subsection*{SQ8}
Suppose that $S$ is a finite set of positive integers. If the greatest integer in $S$ is removed from $S$, then the average value (arithmetic mean) of the integers remaining is 32. If the least integer in $S$ is also removed, then the average value of the integers remaining is 35. If the greatest integer is then returned to the set, the average value of the integers rises to 40. The greatest integer in the original set $S$ is 72 greater than the least integer in $S$.\\
What is the average value of all the integers in the set $S$?

\begin{enumerate}[label=\choice{\Alph*}, leftmargin=2cm]
\item 36.2
\item 36.4
\item 36.6
\item 36.8
\item 37
\end{enumerate}

\subsection*{SQ9}
Hiram's algebra notes are 50 pages long and are printed on 25 sheets of paper; the first sheet contains pages 1 and 2, the second sheet contains pages 3 and 4 and so on. One day he leaves his notes on the table before leaving for lunch, and his roommate decides to borrow some pages from the middle of the notes. When Hiram comes back, he discovers that his roommate has taken a consecutive set of sheets from the notes and that the average (mean) of the page numbers on all remaining sheets is exactly 19.\\
How many sheets were borrowed?

\begin{enumerate}[label=\choice{\Alph*}, leftmargin=2cm]
\item 10
\item 13
\item 15
\item 17
\item 20
\end{enumerate}

\subsection*{SQ10}
When the mean, median and mode of the list 10, 2, 5, 2, 4, 2, $x$ are arranged in increasing order, they form a non-constant arithmetic progression.\\
What is the sum of all possible real values of $x$?

\begin{enumerate}[label=\choice{\Alph*}, leftmargin=2cm]
\item 3
\item 6
\item 9
\item 17
\item 20
\end{enumerate}

\subsection*{SQ11}
Last year Isabella took 7 math tests and received 7 different scores, each an integer between 91 and 100, inclusive. After each test she noticed that the average of her test scores was an integer. Her score on the seventh test was 95.\\
What was her score on the sixth test?

\begin{enumerate}[label=\choice{\Alph*}, leftmargin=2cm]
\item 92
\item 94
\item 96
\item 98
\item 100
\end{enumerate}

\subsection*{SQ12}
For positive integers $m$ and $n$ such that $m + 10 < n + 1$, both the mean and the median of the set $\{m, m + 4, m + 10, n + 1, n + 2, 2n\}$ are equal to $n$.\\
What is $m + n$?

\begin{enumerate}[label=\choice{\Alph*}, leftmargin=2cm]
\item 20
\item 21
\item 22
\item 23
\item 24
\end{enumerate}

\subsection*{SQ13}
The sum of 49 consecutive integers is $7^5$.\\
What is their median?

\begin{enumerate}[label=\choice{\Alph*}, leftmargin=2cm]
\item $7$
\item $7^2$
\item $7^3$
\item $7^4$
\item $7^5$
\end{enumerate}

\subsection*{SQ14}
The set $\{3, 6, 9, 10\}$ is augmented by a fifth element $n$, not equal to any of the other four. The median of the resulting set is equal to its mean.\\
What is the sum of all possible values of $n$?

\begin{enumerate}[label=\choice{\Alph*}, leftmargin=2cm]
\item 7
\item 9
\item 19
\item 24
\item 26
\end{enumerate}

\subsection*{SQ15}
$A$, $B$, $C$ are three piles of rocks. The mean weight of the rocks in $A$ is 40 pounds, the mean weight of the rocks in $B$ is 50 pounds, the mean weight of the rocks in the combined piles $A$ and $B$ is 43 pounds, and the mean weight of the rocks in the combined piles $A$ and $C$ is 44 pounds.\\
What is the greatest possible integer value for the mean in pounds of the rocks in the combined piles $B$ and $C$?

\begin{enumerate}[label=\choice{\Alph*}, leftmargin=2cm]
\item 55
\item 56
\item 57
\item 58
\item 59
\end{enumerate}

\end{document}
