\documentclass[12pt,a4paper]{article}

% Packages
\usepackage{fontspec}
\usepackage{amsmath}
\usepackage{amssymb}
\usepackage{geometry}
\usepackage{xcolor}
\usepackage{enumitem}
\usepackage{booktabs}
\usepackage{array}
\usepackage{tabularx}
\usepackage[unicode]{hyperref}

% Font Setup - MaruBuri Korean Font
\setmainfont{MaruBuri}[
    UprightFont=*-Regular,
    BoldFont=*-Bold,
    ItalicFont=*-Light,
    BoldItalicFont=*-SemiBold
]
\setsansfont{MaruBuri}[
    UprightFont=*-Regular,
    BoldFont=*-Bold
]
\newfontfamily\hangulfont{MaruBuri}[
    UprightFont=*-Regular
]

% Page Setup
\geometry{
    left=2.5cm,
    right=2.5cm,
    top=2.5cm,
    bottom=2.5cm
}

% Line Spacing
\linespread{1.5}

% Table Padding
\renewcommand{\arraystretch}{1.8}

% Custom Commands
\newcommand{\Z}{\mathbb{Z}}
\newcommand{\Q}{\mathbb{Q}}

\begin{document}

이건 아주 핵심적인 통찰이에요.
핵심은 ``\textbf{1.98은 유리수이고}'', $(6+\sqrt{37})$은 \textbf{대수적 수(algebraic number)}이자 \textbf{이차수체의 단위(unit)} 라는 점입니다.
이 두 세계는 완전히 다른 구조를 가집니다.

아래에서 단계적으로 설명할게요.

\section*{1. $(6+\sqrt{37})$은 ``이차수체 단위(unit of quadratic field)''}

이 수는 $\Q(\sqrt{37})$ 라는 \textbf{이차수체}에 속해요.
그 켤레는 $(6-\sqrt{37})$이고, 둘의 곱은

\[
(6+\sqrt{37})(6-\sqrt{37}) = 36 - 37 = -1.
\]

즉,
\[
\alpha = 6+\sqrt{37},\quad \beta = 6-\sqrt{37},\quad \alpha\beta = -1.
\]

이 성질 때문에 $\alpha$는 \textbf{Pell 방정식 $x^2 - 37y^2 = -1$}의 근과 관련된 수입니다.
이 수의 거듭제곱은 다음 꼴로 전개됩니다:
\[
\alpha^n = A_n + B_n\sqrt{37}, \quad \beta^n = A_n - B_n\sqrt{37},
\]
여기서 $A_n, B_n$은 \textbf{모두 정수}입니다.

그리고 $\alpha^n + \beta^n = 2A_n \in \Z$,
$\beta^n$의 절댓값은 $|6-\sqrt{37}|\approx 0.0827 < 1$.

즉,
\[
\alpha^n = \underbrace{(\text{정수})}_{k_n} - \underbrace{\beta^n}_{\text{매우 작음}},
\]
그래서 항상 정수에 \textbf{엄청 가깝습니다}.
이걸 \textbf{``almost integer''} 현상이라고 부릅니다.

\section*{2. $1.98$은 유리수, 켤레가 없다}

$1.98 = \dfrac{99}{50}$은 유리수라서, 어떤 ``켤레''나 ``대수적 구조''가 없습니다.
즉, $1.98$의 거듭제곱은 단순히
\[
(1.98)^n = \frac{99^n}{50^n}
\]
이고, 이건 \textbf{정수/정수} 꼴이지만 분모 $50^n$이 절대 약분되지 않으므로 정수가 되지 않습니다.

게다가, 소수부를 보면
\[
(1.98)^n \bmod 1
\]
은 규칙성을 잃고 \textbf{거의 균등분포 (equidistributed)} 상태가 됩니다.
즉, 어떤 $n$에서는 우연히 정수 근처에 갈 수 있지만, 그런 일이 반복적이거나 구조적으로 일어나지 않습니다.

\section*{3. 왜 이렇게 다를까? --- ``켤레 구조 vs 단순 거듭제곱''}

\begin{table}[h]
\centering
\small
\begin{tabularx}{\textwidth}{|l|l|X|c|X|X|}
\hline
\textbf{구분} & \textbf{구조} & \textbf{켤레 존재} & \textbf{곱} & \textbf{almost integer} & \textbf{예시} \\
\hline
$(6+\sqrt{37})$ & 이차수체 단위 & O: $(6-\sqrt{37})$ & 1 & 매우 높음 (항상 정수 근처) & $(6+\sqrt{37})$, $(3+2\sqrt{2})$ \\
\hline
$(1.98 = 99/50)$ & 유리수 & X: 없음 & --- & 낮음 (우연적) & $1.5$, $1.98$, $1.01$ \\
\hline
\end{tabularx}
\end{table}

이차수체 단위는 ``켤레의 절댓값이 1보다 작음''을 이용해
큰 항 $\alpha^n$과 작은 항 $\beta^n$의 차이로 \textbf{정수 근접 구조}를 형성합니다.
유리수는 그런 켤레항이 없기 때문에 그런 구조가 전혀 생기지 않습니다.

\section*{4. 일반화된 관점}

이 현상은 Pell 방정식과 직접 연결됩니다:
\[
x^2 - Dy^2 = \pm 1.
\]
그 해 $(x_1,y_1)$가 있으면,
$\alpha = x_1 + y_1\sqrt{D}$의 거듭제곱은 항상
\[
\alpha^n = A_n + B_n\sqrt{D}, \quad A_n, B_n \in \Z.
\]
따라서 $\alpha^n$의 정수 부분은 언제나 $A_n$이고,
오차는 $|B_n\sqrt{D} - (\text{정수})|$ 형태로 작아집니다.

이게 바로 ``정수에 점점 가까워지는 대수적 수''의 수학적 구조입니다.

\vspace{1em}

요약하면:

\begin{quote}
• $(6+\sqrt{37})$은 \textbf{Pell 방정식으로부터 오는 대수적 단위}라서
$\alpha^n$이 정수에 지수적으로 가까워집니다.

• $1.98$은 \textbf{단순 유리수}라서
그런 대칭적 구조가 전혀 없고, 정수 근접은 우연적입니다.
\end{quote}

\vspace{1em}

원하시면 제가 ``almost integer''를 만드는 수들의 일반형 $(a+b\sqrt{d})$의 조건을 수식으로 정리하고, $a,b,d$가 어떤 관계를 가져야 하는지도 단계적으로 보여드릴까요?
(이건 Pell 이론과 Diophantine 근사에 바로 연결됩니다.)

\end{document}
