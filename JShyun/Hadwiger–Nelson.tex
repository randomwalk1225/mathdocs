\documentclass[12pt,a4paper]{article}

% Packages
\usepackage{fontspec}
\usepackage{amsmath}
\usepackage{amssymb}
\usepackage{geometry}
\usepackage{xcolor}
\usepackage{enumitem}
\usepackage{booktabs}
\usepackage{array}
\usepackage{tabularx}
\usepackage[unicode]{hyperref}
\usepackage{emoji}

% Font Setup - MaruBuri Korean Font
\setmainfont{MaruBuri}[
    UprightFont=*-Regular,
    BoldFont=*-Bold,
    ItalicFont=*-Light,
    BoldItalicFont=*-SemiBold
]
\setsansfont{MaruBuri}[
    UprightFont=*-Regular,
    BoldFont=*-Bold
]
\newfontfamily\hangulfont{MaruBuri}[
    UprightFont=*-Regular
]

% Page Setup
\geometry{
    left=2.5cm,
    right=2.5cm,
    top=2.5cm,
    bottom=2.5cm
}

% Line Spacing
\linespread{1.5}

% Table Padding
\renewcommand{\arraystretch}{1.8}

% Custom Commands
\newcommand{\Z}{\mathbb{Z}}
\newcommand{\Q}{\mathbb{Q}}

\begin{document}

\title{Hadwiger--Nelson 문제와 모저 스핀들}
\date{}
\maketitle

다음 명제를 보이겠습니다.

\begin{quote}
평면의 모든 점을 빨강--하양--파랑 3색 중 하나로 칠한다고 하자. 그러면 반드시 거리가 1인 같은 색 점쌍이 존재한다.
\end{quote}

핵심 아이디어는 \textbf{3색으로는 절대 색칠될 수 없는 ``단위거리 그래프''를 평면 위에 유한 개의 점만으로 만들어} 모순을 내는 것입니다. 가장 유명한 예가 7개의 꼭짓점으로 이루어진 \textbf{모저 스핀들(Moser spindle)} 입니다. 이 그래프의 모든 변 길이는 1이며, 어떤 3색 칠도 허용하지 않습니다. (즉, 이 그래프의 색수는 4 이상입니다.)

\section*{구성(기하적 배치)}

\begin{enumerate}
\item 한 변의 길이가 1이고 예각이 $60^\circ$ 인 \textbf{마름모}(모든 변=1)를 두 개 준비합니다. 마름모의 짧은 대각선 길이도 1입니다(계산: $d=\sqrt{2-2\cos60^\circ}=1$).

\item 두 마름모가 \textbf{꼭짓점 하나를 공유}하도록 배치합니다(이 점을 $O$라 하자). 각 마름모의 네 꼭짓점들은 모두 서로 단위거리 간선들로 연결됩니다(변 4개 + 짧은 대각선 1개씩).

\item 두 마름모의 ``바깥쪽'' 꼭짓점 두 개(각각을 $A$, $B$)가 \textbf{서로도 거리가 1}이 되도록 배치합니다. 이로써 꼭짓점은 총 7개, 단위거리 간선은 11개인 모저 스핀들이 완성됩니다.
\end{enumerate}

\section*{색 강제(chasing) 논증}

이제 ``거리가 1인 같은 색 쌍이 없다''고 가정하고 모순을 이끌어냅니다.

\begin{itemize}
\item 공유 꼭짓점 $O$의 색을 $c(O)$라 하자.

\item $60^\circ$ 마름모 하나를 보십시오. 마름모의 성질상 \textbf{짧은 대각선의 양 끝}도 서로 거리가 1입니다. 따라서 그 두 끝점은 서로 다른 색이어야 하고, 또 각각은 $O$와도 거리가 1이므로 $c(O)$가 될 수 없습니다.

이 구조는 마름모 안에서 \textbf{대칭을 따라 색이 ``강제''}되는 효과를 냅니다:

\begin{itemize}
\item $O$의 이웃 둘은 $c(O)$와 다른 두 색을 하나씩 가져야 하고,
\item 그 결과 마름모의 ``바깥쪽 끝''은 다시 $c(O)$로 \textbf{강제}됩니다.
\end{itemize}

\item 같은 논리를 공유 꼭짓점 $O$를 기준으로 \textbf{두 번째 마름모}에도 적용하면, 그 마름모의 바깥쪽 끝도 역시 $c(O)$로 강제됩니다.

\item 그런데 구성에서 두 마름모의 바깥쪽 끝점 $A$와 $B$는 \textbf{서로의 거리도 1}이었습니다.

위 강제 결과로 $c(A)=c(B)=c(O)$가 되어 \textbf{거리가 1인 같은 색 쌍} $(A,B)$가 생깁니다. 이는 ``단위거리 동일색 쌍이 없다''는 가정과 모순입니다.
\end{itemize}

따라서 어떤 3색 칠을 하더라도 평면에는 \textbf{반드시} 거리가 1인 같은 색 점쌍이 존재합니다. $\square$

\section*{코멘트 (배경과 응용)}

\begin{itemize}
\item 위 사실은 유명한 Hadwiger--Nelson 문제(``평면의 단위거리 그래프 색수는 몇인가?'')에서 \textbf{3색은 불가능}함을 보여 주는 표준적인 유한 증거(모저 스핀들)에 해당합니다. 즉, 평면을 단위거리 충돌 없이 칠하려면 \textbf{최소 4색}이 필요합니다.

\item 수업/세미나 활용 시에는 좌표로 스핀들을 한 번 배치해 보고(예: 한 변 1의 정삼각형 격자 위에서 두 마름모를 공유 꼭짓점으로 붙인 뒤, 바깥 꼭짓점 간 거리가 1이 되게 회전‧이동) 위의 \textbf{색 강제(chasing) 규칙}을 도식으로 확인시키면 이해가 빠릅니다.
\end{itemize}

\end{document}
