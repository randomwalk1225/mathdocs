\documentclass[12pt,a4paper]{article}
\usepackage{fontspec}
\usepackage{amsmath,amssymb}
\usepackage{enumitem}
\usepackage{geometry}
\usepackage{titlesec}
\usepackage{array}
\usepackage{booktabs}
\usepackage[colorlinks=true,linkcolor=blue,urlcolor=blue]{hyperref}

\geometry{margin=2.5cm}

% Font settings for XeLaTeX
\setmainfont{Times New Roman}
\setsansfont{Arial}

% Title formatting
\titleformat{\section}{\Large\bfseries}{\thesection}{1em}{}
\titleformat{\subsection}{\large\bfseries}{\thesubsection}{1em}{}

% Custom command for answer choices
\newcommand{\choice}[1]{{\upshape (#1)}}

\title{\textbf{TMUA Chapter 3 - Quiz 2:\\Basis of Logic Practices P03}}
\author{}
\date{}

\begin{document}

\maketitle

\noindent\textbf{Time Allowed:} 60 minutes\\
\textbf{Number of Questions:} 17\\
\textbf{Difficulty:} $\star\star\star\star$

\vspace{1em}
\hrule
\vspace{1em}

\section*{Practice Questions}

\subsection*{Q1}
$a$, $b$ and $c$ are real numbers.

Given that $ab = ac$, which of the following statements must be true?

\begin{itemize}
\item[I] $a = 0$
\item[II] $b = 0$ or $c = 0$
\item[III] $b = c$
\end{itemize}

\begin{itemize}[label=\choice{\Alph*}]
\item none of them
\item I only
\item II only
\item III only
\item I and II only
\item I and III only
\item II and III only
\item I, II and III
\end{itemize}

\subsection*{Q2}
For any real numbers $a$, $b$ and $c$ where $a \geq b$, consider these three statements:

\begin{enumerate}
\item $-b \geq -a$
\item $a^2 + b^2 \geq 2ab$
\item $ac \geq bc$
\end{enumerate}

Which of the statements 1, 2, and 3 must be true?

\begin{itemize}[label=\choice{\Alph*}]
\item none of them
\item 1 only
\item 2 only
\item 3 only
\item 1 and 2 only
\item 1 and 3 only
\item 2 and 3 only
\item 1, 2 and 3
\end{itemize}

\subsection*{Q3 [TMUA, 2018S2Q8]}
The diagram shows an example of a mountain profile.

This consists of upstrokes which go upwards from left to right, and downstrokes which go downwards from left to right. The example shown has six upstrokes and six downstrokes. The horizontal line at the bottom is known as sea level.

A mountain profile of order $n$ consists of $n$ upstrokes and $n$ downstrokes, with the condition that the profile begins and ends at sea level and never goes below sea level (although it might reach sea level at any point). So the example shown is a mountain profile of order 6.

Mountain profiles can be coded by using U to indicate an upstroke and D to indicate a downstroke. The example shown has the code UDUUUDUDDUDD. A sequence of U's and D's obtained from a mountain profile in this way is known as a valid code.

Which of the following statements is/are true?

\begin{itemize}
\item[I] If a valid code is written in reverse order, the result is always a valid code.
\item[II] If each U in a valid code is replaced by D and each D by U, the result is always a valid code.
\item[III] If U is added at the beginning of a valid code and D is added at the end of the code, the result is always a valid code.
\end{itemize}

\begin{itemize}[label=\choice{\Alph*}]
\item none of them
\item I only
\item II only
\item III only
\item I and II only
\item I and III only
\item II and III only
\item I, II and III
\end{itemize}

\subsection*{Q4 [TMUA, 2020S2Q19]}
Nine people are sitting in the squares of a 3 by 3 grid, one in each square, as shown. Two people are called neighbours if they are sitting in squares that share a side. (People in diagonally adjacent squares, which only have a point in common, are not called neighbours.)

Each of the nine people in the grid is either a truth-teller who always tells the truth, or a liar who always lies.

Every person in the grid says: 'My neighbours are all liars'.

Given only this information, what are the smallest number and the largest number of people who could be telling the truth?

\begin{itemize}[label=\choice{\Alph*}]
\item smallest: 1, largest: 4
\item smallest: 2, largest: 4
\item smallest: 2, largest: 5
\item smallest: 3, largest: 4
\item smallest: 3, largest: 5
\item smallest: 4, largest: 4
\item smallest: 4, largest: 5
\item smallest: 5, largest: 5
\end{itemize}

\subsection*{Q5}
For the following statements:

\begin{itemize}
\item P: $\frac{x(x - 2)}{1 - x} > 0$
\item Q: $1 < x < 2$
\end{itemize}

about a real number $x$,

\begin{itemize}[label=\choice{\Alph*}]
\item P implies Q, but Q does not imply P.
\item Q implies P, but P does not imply Q.
\item P implies Q, and Q implies P.
\item P and Q contradict each other.
\end{itemize}

\subsection*{Q6}
Pick a whole number.\\
Add one.\\
Square the answer.\\
Multiply the answer by four.\\
Subtract three.

Which of the following statements are true regardless of which starting number is chosen?

\begin{itemize}
\item[I] The final answer is odd.
\item[II] The final answer is one more than a multiple of three.
\item[III] The final answer is one more than a multiple of eight.
\item[IV] The final answer is not prime.
\item[V] The final answer is not one less than a multiple of three.
\end{itemize}

\begin{itemize}[label=\choice{\Alph*}]
\item I, II, and V.
\item I and IV.
\item II and V.
\item I, III, and V.
\item I and V.
\end{itemize}

\subsection*{Q7}
The fixed positive integers $a$, $b$, $c$, $d$ are such that exactly two of the following four statements are valid:

\begin{enumerate}[label=(\roman*)]
\item $a \leq b < c \leq d$
\item $a + b = c + d$
\item $a = c$ and $b = d$
\item $ad = bc$
\end{enumerate}

You are also told that (ii) and (iv) is not the pair of valid statements. Which of the following must be the pair of valid statements?

\begin{itemize}[label=\choice{\Alph*}]
\item (i) and (ii)
\item (i) and (iii)
\item (i) and (iv)
\item (iii) and (iv)
\end{itemize}

\subsection*{Q8}
There are real numbers $x$, $y$ such that precisely one of the statements (a), (b), (c), (d) is true. Which is the true statement?

\begin{itemize}[label=\choice{\Alph*}]
\item $x \geq 0$
\item $x < y$
\item $x^2 > y^2$
\item $|x| \leq |y|$
\end{itemize}

\subsection*{Q9}
You are given two whole numbers $p$ and $q$, and told that three of the following statements concerning $p$ and $q$ are true and that the other statement is false. Which is the false statement?

\begin{itemize}[label=\choice{\Alph*}]
\item $pq$ is even.
\item $p + q$ is even.
\item $2p + q^2$ is odd.
\item $p^2 + 2q$ is odd.
\end{itemize}

\subsection*{Q10}
Three runners, Friedrich, Gottlieb and Hans had a race. Before the race, a commentator said, ``Either Friedrich or Gottlieb will win.'' Another commentator said, ``If Gottlieb comes second, then Hans will win.'' Another said, ``If Gottlieb comes third, Friedrich will not win.'' And another said, ``Either Gottlieb or Hans will be second.'' In the event, it turned out that all the commentators were correct. In what order did the runners finish?

\begin{itemize}[label=\choice{\Alph*}]
\item Friedrich, Gottlieb, Hans
\item Friedrich, Hans, Gottlieb
\item Hans, Gottlieb, Friedrich
\item Gottlieb, Friedrich, Hans
\item Gottlieb, Hans, Friedrich
\end{itemize}

\subsection*{Q11}
Together, Alan and Bill weigh less than Charlie and Dan. Together, Charlie and Edwina weigh less than Bill and Frances. Which of the following is definitely true?

\begin{itemize}[label=\choice{\Alph*}]
\item Together, Alan and Edwina weigh less than Dan and Frances.
\item Together, Dan and Edwina weigh more than Charlie and Frances.
\item Together, Dan and Frances weigh more than Alan and Charlie.
\item Together, Alan and Bill weigh less than Charlie and Frances.
\item Together, Alan, Bill and Charlie weigh the same as Dan, Edwina and Frances.
\end{itemize}

\subsection*{Q12}
A box of fruit contained twice as many apples as pears. Chris and Lily divided them up so that Chris had twice as many pieces of fruit as Lily. Which one of the following statements is always true?

\begin{itemize}[label=\choice{\Alph*}]
\item Chris took at least one pear.
\item Chris took twice as many apples as pears.
\item Chris took twice as many apples as Lily.
\item Chris took as many apples as Lily took pears.
\item Chris took as many pears as Lily took apples.
\end{itemize}

\subsection*{Q13}
Pierre said, ``Just one of us is telling the truth''.\\
Qadr said, ``What Pierre says is not true''.\\
Ratna said, ``What Qadr says is not true''.\\
Sven said, ``What Ratna says is not true''.\\
Tanya said, ``What Sven says is not true''.

How many of them were telling the truth?

\begin{itemize}[label=\choice{\Alph*}]
\item 0
\item 1
\item 2
\item 3
\item 4
\end{itemize}

\subsection*{Q14}
Isobel: ``Josh is innocent''\\
Genotan: ``Tegan is guilty''\\
Josh: ``Genotan is guilty''\\
Tegan: ``Isobel is innocent''

Only the guilty person is lying; all the others are telling the truth. Who is guilty?

\begin{itemize}[label=\choice{\Alph*}]
\item Isobel
\item Josh
\item Genotan
\item Tegan
\item More information required
\end{itemize}

\subsection*{Q15}
The four statements in the box below refer to a mother and her four daughters. One statement is true, three statements are false.

Who is the mother?

\begin{itemize}
\item Alice is the mother.
\item Carol and Ella are both daughters.
\item Beth is the mother.
\item One of Alice, Diane or Ella is the mother.
\end{itemize}

\begin{itemize}[label=\choice{\Alph*}]
\item Alice
\item Beth
\item Carol
\item Diane
\item Ella
\end{itemize}

\subsection*{Q16}
Each of the Four Musketeers made a statement about the four of them, as follows.

\begin{itemize}
\item d'Artagnan: ``Exactly one is lying.''
\item Athos: ``Exactly two of us are lying.''
\item Porthos: ``An odd number of us is lying.''
\item Aramis: ``An even number of us is lying.''
\end{itemize}

How many of them were lying (with the others telling the truth)?

\begin{itemize}[label=\choice{\Alph*}]
\item one
\item one or two
\item two or three
\item three
\item four
\end{itemize}

\subsection*{Q17}
The Tour de Clochemerle is not yet as big as the rival Tour de France. This year there were five riders, Arouet, Barthes, Camus, Diderot and Eluard, who took part in five stages. The winner of each stage got 5 points, the runner up 4 points and so on down to the last rider who got 1 point. The total number of points acquired over the five stages was the rider's score. Each rider obtained a different score overall and the riders finished the whole tour in alphabetical order with Arouet gaining a magnificent 24 points. Camus showed consistency by gaining the same position in four of the five stages and Eluard's rather dismal performance was relieved by a third place in the fourth stage and first place in the final stage.

Explain why Eluard must have received 11 points in all and find the scores obtained by Barthes, Camus and Diderot.

Where did Barthes come in the final stage?

\vspace{2em}

\noindent\hrule

\vspace{1em}

\noindent\textbf{Last Updated:} 2025-01-11

\end{document}
